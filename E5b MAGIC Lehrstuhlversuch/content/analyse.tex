\section{Stereoskopische Analyse des Krebsnebels}
\label{sec:analyse}

\subsection{Datenselektion und -rekonstruktion}

Abbildung~\ref{fig:analysischain} zeigt den typischen Ablauf einer
stereoskopischen Analyse mit MAGIC Daten. Die hierzu verwendeten Programme sind
eingebettet in das MARS Softwarepaket. In einem ersten Schritt durchlaufen die
Rohdaten der beiden Teleskope eine Vorprozessierung. Dazu gehören die Zeit- und
Ladungskalibrierung der einzelnen Pixel der Kamera mit dem Programm
\textit{sorcerer}, sowie das image cleaning [deutsches Wort...?] und die
Parametrisierung des Kamerabildes mit dem Programm \textit{star}. Im
Anschluss daran werden die Daten der Teleskope durch die Software
\textit{superstar} kombiniert und Stereoparameter für die weitere Analyse
berechnet.
\begin{figure}
  \centering
  \includegraphics[width=0.9\textwidth]{figures/analysischain.png}
  \caption{Ablauf einer typischen stereoskopischen Analyse mit MAGIC Daten.}
  % Referenz: MAGIC wiki
  \label{fig:analysischain}
\end{figure}
In einem nächsten Schritt findet die Datenselektion statt. Der Schritt ist
wichtig, um sicherzustellen, dass die verwendeten Daten vergleichbar und
nicht durch äußere Einflüsse verfälscht sind. Somit dient die Selektion in
erster Linie der Qualitätssicherung der Daten. Für die weitere Analyse werden
nur Daten weiterverwendet, die den folgenden Kriterien genügen:
\begin{itemize}
  \item ...
  \item ...
  \item ...
\end{itemize}
[...Begründung für die Wahl der Kriterien...]

In einem nächsten Schritt wird das Programm \textit{coach} (Compressed Osteria
Alias Computation of the Hadroness parameter) verwendet, um Modelle für die
\begin{enumerate}
  \item Separation von Gamma- und Hadronenstrahlung
  \item Energieabschätzung
  \item Positionsrekonstruktion
\end{enumerate}
aufzustellen. Dies ist notwendig, um [...]. Zur Klassifikation wird ein
\textit{random forest} verwendet. Dieser wird trainiert mit einem Teil des
Datensatzes und einem Monte Carlo generierten Datensatz, der auf Grundlage eines
Modells generiert wurde, welches durch Gammastrahlung induzierte Teilchenschauer
beschreibt. Um die Trennkraft des Klassifizierers zu testen wird der Monte
Carlo Datensatz zu Beginn in einen Test- und einen Trainingsdatensatz
aufgeteilt. Somit ist es möglich mit einem Teil des Datensatzes den
Klassifizierer zu trainieren und diesen dann auf dem Testdatensatz zu testen.
In der durchgeführten Analyse wurde der Klassifizierer mit folgenden
Einstellungen trainiert:
\begin{itemize}
  \item ...
  \item ...
  \item ...
\end{itemize}
[...Erklärung der Parameter...]
Für die Energierekonstruktion werden sogenannte \textit{lookup tables}
generiert, durch die eine Zuordnung erfolgt. [...?]

Nachdem die Modelle erstellt wurden, werden diese mit dem Programm
\textit{melibea} auf den vollständigen Datensatz angewendet. Mit dem nun fertig
präparierten Datensatz können nun im letzten Analyseschritt Physikresultate
produziert werden.

\subsection{Ergebnisse der Analyse}

\begin{figure}
  \centering
  \includegraphics[width=0.8\textwidth]{figures/caspar_flux_skymap.pdf}
  \caption{Eine Caption.}
\end{figure}

\begin{figure}
  \centering
  \includegraphics[width=0.8\textwidth]{figures/odie_thetasquared.pdf}
  \caption{Eine Caption.}
\end{figure}

\begin{figure}
  \centering
  \includegraphics[width=0.8\textwidth]{figures/flute_lichtkurve.pdf}
  \caption{Eine Caption.}
\end{figure}

\begin{figure}
  \centering
  \includegraphics[width=0.8\textwidth]{figures/combunfold_energyspectrum.pdf}
  \caption{Eine Caption.}
\end{figure}
