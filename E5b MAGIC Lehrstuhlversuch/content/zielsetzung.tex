\section{Zielsetzung}
\label{sec:zielsetzung}

Ziel des Versuchs ist die Untersuchung des Krebsnebels mit Daten der MAGIC
Teleskope. Der Krebsnebel (NGC 1952) ist ein Supernovaüberrest innerhalb der
Milchstraße im Sternbild Stier und besteht zum größten Teil aus Wasserstoff und
ionisiertem Helium. Entdeckt im 18. Jahrhundert ist der Krebsnebel bis heute
Gegenstand vieler astronomischer Untersuchungen. Im Bereich bis
\SI{500}{\kilo\electronvolt} ist der Nebel eine der stärksten Quellen für
Gammastrahlung.

Im Folgenden werden zunächst kurz die beiden MAGIC Teleskope vorgestellt. Daran
anschließend wird der typische Ablauf einer MAGIC Analyse erklärt und die
verwendeten Analyseparameter aufgeführt. Als Ergebnisse werden die Ausdehnung
und Signifikanz der Quelle, die Lichtkurve, sowie die entfaltete spektrale
Energieverteilung bestimmt. Dabei wird insbesondere auf die Beschreibung und
Motivation der einzelnen Analyseschritte eingegangen.
