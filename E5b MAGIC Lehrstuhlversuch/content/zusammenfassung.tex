\section{Zusammenfassung}
\label{sec:zusammenfassung}

Zusammenfassend lässt sich sagen, dass die Untersuchung des Krebsnebels mit Daten
der MAGIC Teleskope selbst bei geringer Messzeit von nur etwa 12 Stunden sehr gute
Ergebnisse liefert. Der berechnete Fluss der Quelle, der in der Lichtkurve
aufgetragen ist, ist im Rahmen der bestimmten Unsicherheiten für alle
Messzeitpunkte in guter Übereinstimmung mit den Referenzwerten. Abweichungen
können erklärt werden durch veränderte äußere Bedingungen, wie zum Beispiel
wechselndem Wetter, die trotz sorgfältiger Datenselektion nie ganz kompensiert
werden können. Wie zu erwarten, kann der Krebsnebel mit einer sehr hohen
Signifikanz von $\num{71.5}\sigma$ beobachtet werden. Auch die Entfaltung der
spektralen Energieverteilung liefert Ergebnisse die in guter Übereinstimmung mit
Referenzwerten sind. Aufgrund seiner großen Helligkeit und niedrigen
Variabilität eignet sich der Krebsnebel gut dazu, neue Teleskope oder Software
auf Performance zu testen und zu kalibrieren.
