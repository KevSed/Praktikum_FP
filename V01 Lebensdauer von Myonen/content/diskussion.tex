\section{Diskussion}
\label{sec:diskussion}

\subsection{Kalibrierung und vorbereitende Messungen}

Die Bestimmung der optimalen Verzögerungszeit liefert die in
Abbildung~\ref{fig:verzoegerung} dargestellten Zählraten.
Bei~$T_{\symup{VZ}}=\SI{5.5}{\nano\second}$ wird ein recht deutliches Maximum
gemessen, obwohl eigentlich aufgrund der Breite des Diskriminatorimpulses ein
Plateau erwartet wird. Jedoch liegen die Zählraten insgesamt eng beieinander
und sind mit Fehlern behaftet, deren Größe der Schwankung zwischen den
verschiedenen Messwerten nahekommt. Somit kann die Nichtbeobachtung eines
Plateaus auch das Resultat statistischer Ungenauigkeiten sein.

Die Raten werden durch Zählen der Impulse in einem~\SI{20}{\second}-Intervall
bestimmt. Eine längere Zählzeit würde zu einem präziseren Ergebnis führen.
Zur Zeitmessung wird eine von Hand bediente Stoppuhr verwendet. Dies führt
aufgrund der menschlichen Reaktionszeit zu weiteren Fehlern, die das Ergebnis
beeinflussen.

Die Kalibrierung der Zeitachse des Vielkanalanalysators erfolgt mit Hilfe einer
linearen Regression. Die Unsicherheiten der berechneten Steigung ist
mit~\SI{0.02}{\percent} sehr klein. Dies spricht für eine gute Qualität der
Kalibrierung.

\subsection{Bestimmung der mittleren Lebensdauer eines Myons}

Im Versuch wird die mittlere Lebensdauer eines Myons zu
%
\begin{equation}
  \tau=\SI{2.121(17)}{\micro\second}
\end{equation}
%
bestimmt. Der Literaturwert~\cite[14]{pdg} beträgt
%
\begin{equation}
  \tau_{\symup{lit}}=\SI{2.1969811(2)}{\micro\second}
\end{equation}
%
woraus sich eine relative Abweichung von~\SI{3.5}{\percent} ergibt. Der
Literaturwert liegt innerhalb der~$5\sigma$-Umgebung des Messwertes.

Der gemessene Untergrund, der sich aus der Ausgleichsrechnung der Messwerte
ergibt, beträgt~$U_0=\num{3.75(24)}$ Ereignisse pro Kanal. Der erwartete
Untergrund wird zu~$U_0=\num{5.221(3)}$ Ereignisse pro Kanal bestimmt.
Somit liegt der erwartete Wert knapp außerhalb der~$6\sigma$-Umgebung des
gemessenen Wertes. Erwartung und Messung des Untergrundes passen nicht sehr gut
zusammen.

Insgesamt werden~\num{50068} Stoppimpulse bei~\num{4179917} Startimpulsen
vom Zählwerk gemessen. Die Addition aller Impulse, die der Vielkanalanalysator
an den Rechner geleitet hat, ergibt jedoch nur~\num{42543} Impulse. Die
beobachtete Differenz von~\num{7525} Impulsen lässt auf Fehler am
Vielkanalanalysator oder an der Übertragungsleitung schließen. Es wird vermutet,
dass im Bereich kleiner Zeitdifferenzen (Kanal~\num{0} bis~\num{2}) viele
Impulse vom Vielkanalanalysator nicht registriert werden. Eine Abschätzung der
für diese Kanäle zu erwartenden Impulsanzahl liefert mit Hilfe von
Gleichung~\eqref{eq:fitfunktion} ungefähr~\num{2140} Impulse, womit ein Teil
der fehlenden Impulse erklärt werden könnte.
