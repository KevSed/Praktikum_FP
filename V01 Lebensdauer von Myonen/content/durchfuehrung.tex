\section{Durchführung}
\label{sec:durchführung}

Abbildung~\ref{fig:aufbau} zeigt schematisch den Aufbau der Messapparatur, sowie
die verwendete Schaltung. An dem oben bereits beschriebenen Szintillator
befinden sich an beiden Seiten die Sekundärelektronenvervielfacher (SEV). Diese
wandeln die von den Photonen erzeugten Elektronen in Spannungssignale um. Dazu
werden sie mit Hochspannung versorgt. Die so erzeugten Signale werden über
Verzögerungsleitungen von beiden Seiten des Szintillators an zwei
Diskriminatoren gelegt. Die Verzögerungsleitungen dienen dabei dem Ausgleich der
Zeitspannen zwischen Erzeugung eines Photons und Ankunft des dazugehörigen
elektrischen Signals am Diskriminator. So lässt sich die Zahl der ermittelten
Ereignisse maximieren. Der Diskriminator besitzt eine Spannungsschwelle, ab
welcher Signale als solche gewertet werden. Dies sorgt dafür, dass Schwankungen
durch Detektorrauschen nicht als Signale gewertet werden. Signale, die als
solche gewertet werden, werden darüber hinaus in NIM Logik entsprechende Signale
gleicher Amplitude gewandelt. Diese Signale erreichen anschließend die
Koinzidenz-Schaltung. Diese sorgt dafür, dass Impulse, die in einem gewissen
Zeitintervall~$\upD t$ die Schaltung erreichen als gleichzeitig gelten. Solche
Impulse werden anschließend an die beiden AND-Gatter weitergeleitet.

Um die Zeit zwischen dem Eindringen der Myonen in den Detektor und deren
anschließenden Zerfall zu bestimmen, müssen die dabei detektierten Lichtblitze
in \enquote{Start}- und \enquote{Stopp}-Signale einer Zeitmessung umgewandelt
werden. Weil nur wenige Myonen zwei Lichtimpulse im Szintillator erzeugen, also
so niederenergetisch sind, dass sie im Szintillator zerfallen, ist mit deutlich
mehr Start- als Stopp-Impulsen zu rechnen. Die im unteren Teil des Diagramms
dargestellte Schaltung dient dem Zweck dennoch möglichst nur die physikalisch
sinnvollen Zeitintervalle aufzunehmen. Dazu wird die \enquote{Wartezeit} bis zum
Stopp-Impuls begrenzt. Die Ausgangssignale der Koinzidenzschaltung gelangen an
die beiden AND-Gatter und aktivieren so den START-Eingang des
\textit{time-amplitude-converter} (TAC), sowie den Suchzeitgeber. Der TAC sorgt
dafür, dass die gemessenen Zeiträume digitalisiert und in verschiedenen Kanälen
gespeichert werden. Der Suchzeitgeber ist eine so genannte Kippstufe. Diese
Kippstufe kann durch ein elektrisches Signal \enquote{angestoßen} werden und
legt dann für eine begrenzte Zeit~$T_{\symup{S}}$ ein Signal an das zweite
AND-Gatter. Zerfällt das Myon in diesem Zeitraum im Tank und sendet somit die
Koinzidenzschaltung einen Impuls, so gelangt dieser zum Stopp-Eingang des TAC
und ein Messwert wird aufgezeichnet. Nach Ablauf dieses Zeitraums beendet die
Kippstufe ihr Signal und die Schaltung begibt sich wieder in den Anfangszustand.
Signale die nach diesem Zeitraum eintreffen werden also wieder als Start-Impulse
registriert. Die Länge dieses Zeitraumes sollte daher so gewählt werden, dass
die Dauer von Start-Impuls bis zum Zerfall eines Myons deutlich kleiner ist
(damit alle realen Signale auch eingefangen werden) allerdings kleiner als die
erwartete Zeitspanne in welcher zwei zufällige Myonen in den Szintillator
vordringen, sodass diese möglichst nicht als der Zerfall eines Myons
interpretiert werden.

\begin{figure}[htb]
  \centering
  \includegraphics[width=0.9\textwidth]{figures/versuchsaufbau_adapt.pdf}
  \caption{Schematischer Aufbau der Versuchsapparatur und der damit verbundenen Schaltung (adaptiert)~\cite{V01}.}
  \label{fig:aufbau}
\end{figure}

\subsection{Durchführung}

Die Durchführung dieses Versuches gliedert sich in zwei Teile: zunächst wird die
Schaltung aufgebaut und die einzelnen Komponenten getestet. Die Schaltung wird
so eingestellt, dass die gemessene Rate maximal wird und anschließend eine
Kalibrationsmessung durchgeführt. Ist die Messapparatur betriebsbereit, wird die
Messung gestartet und im zweiten Teil die Daten aufgenommen.

Zur Einstellung der Messschaltung sind zunächst die Diskriminatoren
einzustellen. Diese werden mit Hilfe von Impulsen und einem Oszilloskop so
eingestellt, dass die Breiten (also Dauern) der Impulse, die den Diskriminator
verlassen einheitlich sind. Anschließend sind die zeitlichen Differenzen der
Signale der verschiedenen SEVs auszugleichen. Dazu werden diese über
Verzögerungsleitungen relativ zueinander verzögert. Es wird eine feste
Verzögerung für eine Leitung festgelegt und relativ dazu die Verzögerung der
anderen variiert. Das Ziel ist hierbei eine möglichst hohe Zählrate zu
generieren, also die Impulse möglichst übereinander zu legen. Dazu wird eine
Messreihe der Zählrate in Abhängigkeit der relativen Verzögerung der beiden
Kanäle zueinander durchgeführt.

Bevor die Messung gestartet werden kann, muss noch die Kalibrierung der
Zeitkanäle des TAC vorgenommen werden. Dazu werden von einem
Doppelimpulsgenerator Impulse verschiedener zeitlicher Abstände an den TAC
geleitet und anschließend am Rechner gespeichert. Durch Vermessen verschiedener
Zeitintervalle lassen sich den 512 Kanälen die zugehörigen Zeitintervalle
zuordnen. Nach dieser Kalibrierung kann die Messung gestartet werden. Die
Schaltung darf dabei nach der Kalibrierung nicht mehr verändert werden. Am
Rechner wird das Messprogramm gestartet und über eine Dauer
von~\SI{1520165}{\second} Daten genommen.
