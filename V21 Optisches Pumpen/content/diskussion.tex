\section{Diskussion}
\label{sec:diskussion}
%
Die vertikale Komponente des Erdmagnetfeldes wird
zu~$B_{\bot}=\SI{37.55}{\micro\tesla}$ bestimmt. Damit liegt der Wert um
ungefähr~$\SI{17}{\percent}$ unterhalb des
Literaturwertes~$B_{\bot\text{,Lit}}=\SI{45.07}{\micro\tesla}$~\cite{igrf}. Aus
den Regressionsgeraden in Abbildung~\ref{fig:magnetfeld} ergibt sich die
horizontale Komponente des Erdmagnetfeldes als y-Achsenabschnitt
zu~$B_{\parallel}=\SI{15.7(5)}{\micro\tesla}$. Dieser Wert weicht
ungefähr~$\SI{19}{\percent}$ nach unten vom
Literaturwert~$B_{\parallel\text{,Lit}}=\SI{19.33}{\micro\tesla}$~\cite{igrf}
ab. Mögliche Fehlerquellen sind Störfelder, die durch andere Experimente im Raum
verursacht werden.

Im zweiten Teil werden die Kernspins der Rubidiumisotope
zu~$I_{87}=\num{1.520(16)}$ und~$I_{85}=\num{2.505(13)}$ bestimmt. Damit liegen
die theoretischen Werte~$\sfrac{3}{2}$ bzw.~$\sfrac{5}{2}$ innerhalb von zwei
bzw. einer Standardabweichung der experimentell bestimmten Werte. Die Messung ist
folglich als sehr präzise einzuschätzen. Die hohe Genauigkeit der Bestimmmung
der Kernspins bedeutet eine ebenso gute Bestimmung der
Landé-Faktoren.

Das Isotopenverhältnis innerhalb der Dampfzelle wird
zu~$\sfrac{N\left(\ce{^85Rb}\right)}{N\left(\ce{^87Rb}\right)}=\num{2.0(2)}$
berechnet. Natürliches Rubidium hat eine Isotopenverhältnis von~$\num{0.67}$. Es
ist folglich davon auszugehen, dass das vorligende Rubidium mit~$\ce{^85Rb}$
angereichert wurde.

Die Untersuchung des quadratischen Zeeman-Effektes liefert einen Unterschied von
drei Größenordnungen für den Beitrag des linearen und des quadratischen Terms
zur Zeeman-Energie. Der quadratische Term kann somit guten Gewissens in den
Berechnungen vernachlässigt werden.

Das im letzten Auswertungsteil aus den Parametern der angepassten
Hyperbelfunktion bestimmte Verhältnis~$\sfrac{b_{85}}{b_{87}}$
beträgt~$\num{1.42(5)}$. Der theoretische Wert beträgt~$\num{1.5}$~\cite{V21}
und liegt somit innerhalb von zwei Standardabweichungen um den experimentell
bestimmten Wert.
