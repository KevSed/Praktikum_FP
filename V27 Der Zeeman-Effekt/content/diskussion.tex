\section{Diskussion}
\label{sec:diskussion}

Die Ergebnisse der Auswertung, die theoretisch erwarteten Werte, sowie die Abweichungen zwischen diesen Größen sind in Tabelle~\ref{tab:ergebnisse} aufgeführt. Die statistischen Abweichungen der Lande-Faktoren sind hierbei nicht wirklich aussagekräftig, da ansonsten keine weiteren Fehler bestimmt werden können und die Fortpflanzung dieser daher nicht berücksichtigt werden kann.
%
\begin{table}[H]
    \centering
    \caption{Ergebnisse der Auswertung der Lande-Faktoren und theoretische Erwartungen.}
    \begin{tabular}{cccc}
        \toprule
    {Polarisation}  & {$g_{\text{exp}}$}  & {$g_{\text{theo}}$} & {Abweichung} \\
		\midrule
	  $\sigma_\text{rot}$ & 0.91 & 1 & $\SI{9}{\percent}$ \\
    $\sigma_\text{blau}$ & 1.4 & 1.75 & $\SI{20}{\percent}$ \\
    $\pi_\text{blau}$ & 0.55 & 0.5 & $\SI{10}{\percent}$ \\
		\bottomrule
	\end{tabular}
    \label{tab:ergebnisse}
\end{table}
%
Mit Abweichungen von etwa $\SI{10}{\percent}$ sind die bestimmten Lande-Faktoren für die rote $\sigma$-Linie und die blaue $\pi$-Linie in einem akzeptablen Rahmen. Beide Größen weichen nach unten ab, sodass auf einen systematischen Fehler geschlossen werden kann. Ausschlaggebend hierbei ist sicherlich die Auswertung. Durch das Ablesen nach Augenmaß ist eine systematische Fluktuation nicht vermeidbar. Der Lande-Faktor für die blaue $\sigma$-Linie weicht mit $\SI{20}{\percent}$ etwas stärker nach oben ab. Da die Bilder allerdings auch während ihrer Aufnahme durch Verstellen der Kamera, sowie der Apparatur veränderten Randbedingungen unterlagen, ist dieser Unterschied nicht wirklich einzuordnen.
