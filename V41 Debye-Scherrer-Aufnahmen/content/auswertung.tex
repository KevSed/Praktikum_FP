\section{Auswertung}
\label{sec:auswertung}

\subsection{Fehlerrechnung}

Die in der Auswertung verwendeten Mittelwerte mehrfach gemessener Größen sind
gemäß der Gleichung
%
\begin{equation}
    \bar{x}=\frac{1}{n}\sum_{i=1}^n x_i
    \label{eq:mittelwert}
\end{equation}
%
bestimmt. Die Standardabweichung des Mittelwertes ergibt sich dabei zu
%
\begin{equation}
    \symup{\Delta}\bar{x}=\sqrt{\frac{1}{n(n-1)}\sum_{i=1}^n\left(x_i-\bar{x}\right)^2}.
    \label{eq:standardabweichung}
\end{equation}
%
Resultiert eine Größe über eine Gleichung aus zwei anderen fehlerbehafteten
Größen, so berechnet sich der Gesamtfehler nach der Gaußschen
Fehlerfortpflanzung zu
%
\begin{equation}
    \symup{\Delta}f(x_1,x_2,...,x_n)=\sqrt{\left(\frac{\partial f}{\partial x_1}\symup{\Delta}x_1\right)^2+\left(\frac{\partial f}{\partial x_2}\symup{\Delta}x_2\right)^2+ \dotsb +\left(\frac{\partial f}{\partial x_n}\symup{\Delta}x_n\right)^2}.
    \label{eq:fehlerfortpflanzung}
\end{equation}
%
Alle in der Auswertung angegebenen Größen sind stets auf die erste signifikante
Stelle des Fehlers gerundet. Setzt sich eine Größe über mehrere Schritte aus
anderen Größen zusammen, so wird erst am Ende gerundet, um Fehler zu vermeiden.
Zur Auswertung wird die Programmiersprache \texttt{python (Version 3.5.1)} mit
den Bibliothekserweiterungen \texttt{numpy}~\cite{numpy},
\texttt{scipy}~\cite{scipy} und \texttt{matplotlib}~\cite{matplotlib} verwendet.


\subsection{Vorgehensweise}

In einem ersten Schritt werden die Netzebenen verschiedener elementarer Gitter
bestimmt, deren Reflexe nicht verschwinden. Dazu wird die Gleichung für die
Strukturamplitude~\eqref{eq...} zugrunde gelegt. Durch Variation der
Indices~$(hkl)$ werden die Kombinationen ausgemacht, für die Strukturamplitude
nicht Null wird. In einem zweiten Schritt wird durch Einsetzen von
Gleichung~\eqref{eq:...} in Gleichung~\eqref{eq:...} der Ausdruck



für die Gitterkonstante aufgestellt. Die Normierung liefert den Zusammenhang



Nun werden die normierten Wurzeln für die zulässigen Werte von~$N$ berechnet.
Tabelle~\ref{tab:...} listet die verschiedenen Werte für die verschiedenen
Gittertypen auf.



Es wird nun versucht an Hand der experimentell angefertigten
Debye-Scherrer-Aufnahmen die Quotienten~$d_1/d_i$ zu bestimmen. Durch Vergleich
mit Tabelle~\ref{tab:...} kann dann auf den zugrundeliegenden Gittertyp
geschlossen werden.

Es erweist sich als vorteilhaft, dass der Umfang des
Kameragehäuses~$\SI{360}{\milli\metre}$ beträgt, da die gemessenen Abstände der
aufgenommenen Linien direkt als Winkelmaß~$2\theta$ in Grad übersetzt werden
können. Die Messung der Linienabstände auf dem entwickelten Fotofilm erfolgt mit
Hilfe eines Lineals. Als Ablesefehler wird~\SI{1}{\milli\metre} angenommen. Die
Berechnung der Netzebenenabstände~$d$ erfolgt mit Hilfe der Bragg-Bedingung in
Gleichung~\eqref{eq:...}. Dabei ist die Wellenlänge der Röntgenstrahlung
mit~$\lambda=\SI{0}{\metre}$ gegeben. Der Wert ergibt sich durch Mittelung der
Wellenlänge für die~$K_{\alpha}$- und die~$K_{\beta}$-Linie der verwendeten
[...]-Röntgenquelle.

Nach erfolgter Bestimmung des Gittertyps werden in einem weieren
Auswertungsschritt die Gitterkonstanten der untersuchten Proben bestimmt. Dazu
werden mit Hilfe von Gleichung~\eqref{eq:...} die Gitterkonstanten an Hand der
verschiedenen Messpunkte bestimmt. Durch lineare Regression der Messwerte
gegen~$\cos^2{\theta}$ kann dann der beste Wert für die Gitterkonstante als
$y$-Achsenabschnitt bestimmt werden.

\subsection{Untersuchung der Probe 8}

\subsection{Untersuchung von Salz 2}

Die Untersuchung von Salz 2 liefert die in Tabelle~\ref{tab:...}
aufgeführten Ergebnisse. Es zeigt sich, dass die beobachteten Beugungsreflexe
am besten durch ein $bcc$-Gitter erklärt werden können. Zwar fehlt der zur
$...$-Ebene gehörige Reflex, doch kann dies auch aufgrund hoher Schwärzung dem
schwer auszuwertenden Fotofilm geschuldet sein. Abbildung zeigt die berechneten
Gitterkonstanten aufgtragen gegen~$\cos^2{\theta}$. Die lineare Regression
der Form~$a=m\cdot\cos^2{\theta}+b$ liefert die Parameter

\begin{figure}[h]
  \centering
  \caption{Eine Caption.}
  \begin{tabular}{S[table-format=3(1)]
                  S[table-format=2.1(1)]
                  S[table-format=1.3(3)]
                  S[table-format=1.2(2)]
                  S[table-format=2]
                  S[table-format=2.3]
                  S[table-format=1.2(2)]}
    \toprule
    {$r$}  & {$\theta$} & {$d$} & {$d_1/d_i$} & {$N$} & {$\sqrt{N_i/N_1}$} & {$a$} \\
    \midrule
     31(1) & 15.5(5) & 2.89(9)  & 1       &  2 & 1     & 4.1(1)  \\
     45(1) & 22.5(5) & 2.02(4)  & 1.43(5) &  4 & 1.414 & 4.03(8) \\
     55(1) & 27.5(5) & 1.67(3)  & 1.73(6) &  6 & 1.732 & 4.09(7) \\
     64(1) & 32.0(5) & 1.46(2)  & 1.98(7) &  8 & 2.000 & 4.12(6) \\
     73(1) & 36.5(5) & 1.30(1)  & 2.23(7) & 10 & 2.236 & 4.10(5) \\
     89(1) & 44.5(5) & 1.10(1)  & 2.62(9) & 14 & 2.646 & 4.12(4) \\
    102(1) & 51.0(5) & 0.993(7) & 2.91(9) & 16 & 2.828 & 3.97(3) \\
    106(1) & 53.0(5) & 0.966(6) & 3.0(1)  & 18 & 3.000 & 4.10(3) \\
    114(1) & 57.0(5) & 0.920(5) & 3.1(1)  & 20 & 3.162 & 4.11(2) \\
    123(1) & 61.5(5) & 0.878(4) & 3.3(1)  & 22 & 3.317 & 4.12(2) \\
    133(1) & 66.5(5) & 0.841(3) & 3.4(1)  & 24 & 3.464 & 4.12(2) \\
    145(1) & 72.5(5) & 0.809(2) & 3.6(1)  & 26 & 3.606 & 4.12(1) \\
    \bottomrule
  \end{tabular}
  \label{tab:salz2}
\end{figure}
