\section{Diskussion}
\label{sec:diskussion}

Zusammengefasst lässt die Untersuchung der Probe 8 auf eine Diamant-Struktur
mit einer Gitterkonstanten von~$\SI{5.5(2)}{\angstrom}$ schließen. Es könnte
sich somit um Zinksulfid handeln, welches in einer Diamantstruktur
kristallisiert und eine Gitterkonstante von~$\SI{}{\angstrom}$ besitzt.

Für das Salz 2 wurde die Gitterkonstante zu~$\SI{4.11(3)}{\angstrom}$ bestimmt,
wobei es sich bei dem Gitter am wahrscheinlichsten um eine
Cäsiumchlorid-Struktur handelt. Da Cäsiumchlorid die
Gitterkonstante~$\SI{}{\angstrom}$ besitzt, handelt es sich bei Salz 2
wahrscheinlich um Cäsiumchlorid.

Typische Fehlerquellen des Versuchs sind...
