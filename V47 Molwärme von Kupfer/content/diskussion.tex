\section{Diskussion}
\label{sec:diskussion}
%
Der Literaturwert für die Debye-Temperatur beträgt~$\theta_{\mathrm{D,lit}}=\SI{343}{\kelvin}$~\cite{kittel}.
Es zeigt sich, dass dieser nur um ungefähr~\SI{10}{\kelvin} von~$\theta_{\mathrm{D,2}}$ abweicht, welcher mit Hilfe theoretischer Überlegungen aus Forderung~\eqref{...} bestimmt wurde.
Viel größer ist hingegen die Abweichung des im Versuch gemessenen Wertes~$\theta_{\mathrm{D,1}}$ vom Literaturwert.
Diese beträgt rund~\SI{21.6}{\percent} und könnte dadurch erklärt werden, dass der Temperaturgradient zwischen Rezipient und Probe zwischenzeitlich so groß war, dass Wärmestrahlung die Messung gestört hat.
Eine weitere Fehlerquelle kann dadurch entstanden sein, dass die Probe zu Beginn der Messung nicht kalt genug war.
Durch eine größere Messgenauigkeit im Tieftemperaturbereich kann das Ergebnis optimiert werden.
