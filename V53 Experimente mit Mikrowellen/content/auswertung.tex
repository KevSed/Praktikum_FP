\newpage
\section{Auswertung}
\label{sec:auswertung}

\subsection{Untersuchung eines Reflex-Klystrons}

Tabelle~\ref{tab:V1TabI} zeigt die zur Erstellung eines Modendiagramms aufgenommenen Messwerte dreier verschiedener Moden des Klystrons.
Das zugehörige Diagramm ist in Abbildung~\ref{fig:modenkurve} zu finden.
Die Koeffizienten der dargestellten Parabeln werden gemäß der allgemeinen Vorschrift $A(U)=aU^2+bU+c$ durch eine Ausgleichsrechnung bestimmt und sind in Tabelle~\ref{tab:koeffizienten} zusammengefasst.\footnote{Zur Auswertung wird das Programm \texttt{python (Version 3.4.1)} verwendet.}

\begin{table}[h]
    \centering
    \caption{Daten der Moden.}
    \begin{tabular}{lcSSSr}
        \toprule
		                  &       & {1. Modus}  & {2. Modus}  & {3. Modus}  &       \\
		\midrule
		Reflektorspannung & $U_0$ & \SI{200}{}  & \SI{96}{}   & \SI{120}{}  & [V]   \\
		                  & $U_1$ & \SI{190}{}  & \SI{90}{}   & \SI{110}{}  &       \\
		                  & $U_2$ & \SI{205}{}  & \SI{105}{}  & \SI{124}{}  &       \\
		Amplitude         & $A_0$ & \SI{1.08}{} & \SI{0.92}{} & \SI{1.13}{} & [V]   \\
		Frequenz          & $f_0$ & \SI{9000}{} & \SI{9006}{} & \SI{9003}{} & [MHz] \\
		\bottomrule
	\end{tabular}
    \label{tab:V1TabI}
\end{table}

\begin{figure}[h]
    \centering
    \includegraphics[width=0.85\textwidth]{build/plot_modenkurve.pdf}
    \caption{Drei verschiedene Modenkurven des Klystrons.}
    \label{fig:modenkurve}
\end{figure}

\begin{table}[h]
    \centering
    \caption{Koeffizienten der Modenkurven.}
    \begin{tabular}{cSSS}
        \toprule
		Modus & {$a$}           & {$b$}       & {$c$}          \\
              & {[1/V]}         &             & {[V]}          \\
        \midrule
		1     & \SI{-2.16e-2}{} & \SI{8.53}{} & \SI{-8.41e2}{} \\
        2     & \SI{-1.70e-2}{} & \SI{3.32}{} & \SI{-1.61e2}{} \\
        3     & \SI{-2.83e-2}{} & \SI{6.61}{} & \SI{-3.85e2}{} \\
		\bottomrule
	\end{tabular}
    \label{tab:koeffizienten}
\end{table}

Die Messung bei elektronischer Abstimmung liefert die Werte in Tabelle~\ref{tab:V1TabII}.
Die elektronische Bandbreite $\Delta f$ berechnet sich damit zu:
%
\begin{equation}
    \Delta f=f'-f''=\SI{9015}{\mega\hertz}-\SI{8980}{\mega\hertz}=\SI{35}{\mega\hertz}.
\end{equation}
%
Die elektronische Abstimmempfindlichkeit $A$ beträgt:
%
\begin{equation}
    A=\frac{f'-f''}{U'-U''}=\frac{\SI{9015}{\mega\hertz}-\SI{8980}{\mega\hertz}}{\SI{209}{\volt}-\SI{192}{\volt}}\approx\SI{2.06}{\mega\hertz\per\volt}.
\end{equation}

\begin{table}[h]
    \centering
    \caption{Elektronische Abstimmung.}
    \begin{tabular}{lcSSSr}
        \toprule
		Reflektorspannung & $U_R$ & \SI{200}{}  & \SI{209}{}  & \SI{192}{}  & [V]   \\
		Frequenz          & $f$   & \SI{8998}{} & \SI{9015}{} & \SI{8980}{} & [MHz] \\
		\bottomrule
	\end{tabular}
    \label{tab:V1TabII}
\end{table}

\subsection{Messung von Frequenz, Wellenlänge und Dämpfung}

Die im zweiten Versuchsteil aufgenommenen Messwerte sind in Tabelle~\ref{tab:V2TabI} zu finden.
Die direkte Messung der Frequenz der höchsten Mode mit dem Frequenzmesser ergibt den Wert~$f_{\text{gemessen}}$.
Zur Berechnung der Frequenz aus der Wellenlänge im freien Raum wird zunächst die Wellenlänge $\lambda_g$ im Hohlleiter bestimmt.
Diese ergibt sich als das Doppelte des Abstandes der gemessenen Minima.
Zusammen mit der Abmessung des Hohlleiters und der Formel~\eqref{eq:frequenz} des Theorieteils folgt:
%
\begin{equation}
    f_{\text{berechnet}}=\SI{3e11}{\milli\metre\per\second}\sqrt{\frac{1}{(\SI{47.4}{\milli\metre})^2}+\frac{1}{4\cdot(\SI{22.8}{\milli\metre})^2}}\approx\SI{9129.1}{\mega\hertz}.
\end{equation}

\begin{table}[h]
    \centering
    \caption{Frequenz und Wellenlänge.}
    \begin{tabular}{SSSSSS}
        \toprule
		{$f_{\text{gemessen}}$} & {1. Min}    & {2. Min}      & {$\lambda_g$} & {$a$}       & {$f_{\text{berechnet}}$} \\
        {[MHz]}                 & {[mm]}      & {[mm]}        & {[mm]}        & {[mm]}      & {[MHz]}                  \\
        \midrule
		\SI{9000}{}             & \SI{57.2}{} & \SI{80.9}{}   & \SI{47.4}{}   & \SI{22.8}{} & \SI{9129.1}{}            \\
		\bottomrule
	\end{tabular}
    \label{tab:V2TabI}
\end{table}

Die Ergebnisse zur Dämpfungsmessung sind in Tabelle~\ref{tab:V2TabII} aufgeführt.
Die Mikrometereinstellung wird so gewählt, dass sich die Dämpfung am SWR-Meter wie aufgeführt ergibt.
Dann wird diese mit dem theoretischen Wert, der aus einer Eichkurve am Dämpfungsglied abzulesen ist, verglichen.
Abbildung~\ref{fig:daempfungskurve} zeigt die Werte graphisch aufgetragen.\footnote{Zur Auswertung wird das Programm \texttt{python (Version 3.4.1)} verwendet.}
Die Messergebnisse stimmen in vier Punkten mit den theoretischen Werten überein.

\begin{table}[h]
    \centering
    \caption{Dämpfungsmessung.}
    \begin{tabular}{SSS}
        \toprule
		{SWR-Meter} & {Mikrometereinstellung} & {Theoretischer Wert} \\
        {[dB]}      & {[mm]}                  & {[dB]}               \\
        \midrule
		\SI{0}{}    & \SI{0.00}{}             & \SI{0}{}             \\
        \SI{2}{}    & \SI{1.05}{}             & \SI{3}{}             \\
        \SI{4}{}    & \SI{1.50}{}             & \SI{4.5}{}           \\
        \SI{6}{}    & \SI{1.77}{}             & \SI{6}{}             \\
        \SI{8}{}    & \SI{2.02}{}             & \SI{8}{}             \\
        \SI{10}{}   & \SI{2.29}{}             & \SI{10}{}            \\
		\bottomrule
	\end{tabular}
    \label{tab:V2TabII}
\end{table}

\begin{figure}[h]
    \centering
    \includegraphics[width=0.85\textwidth]{build/plot_daempfungskurve.pdf}
    \caption{Dämpfungskurve.}
    \label{fig:daempfungskurve}
\end{figure}

\subsection{Stehwellen-Messungen}

Die Messwerte zu den drei verschiedenen Methoden der Stehwellenmessung sind in den Tabellen~\ref{tab:V3TabI} bis~\ref{tab:V3TabIII} aufgeführt.
Bei der 3-dB-Methode berechnet sich die Wellenlänge im Hohlleiter als der doppelte Abstand der beiden ausgemessenen Minima.
Das SWR berechnet sich zu:
%
\begin{equation}
	\text{SWR}=\sqrt{1+\frac{1}{\sin^2\left(\frac{\pi(d_1-d_2)}{\lambda_g}\right)}}=\sqrt{1+\frac{1}{\sin^2\left(\frac{\pi(\SI{90.4}{\milli\metre}-\SI{88.4}{\milli\metre})}{\SI{49}{\milli\metre}}\right)}}\approx\SI{7.8}{}
\end{equation}

\begin{table}[h]
    \centering
    \caption{Die SWR-Meter-Methode.}
    \begin{tabular}{lSSSSr}
        \toprule
		Sondentiefe & \SI{3}{}    & \SI{5}{}    & \SI{7}{}    & \SI{9}{}     & [mm] \\
		SWR         & \SI{1.16}{} & \SI{1.58}{} & \SI{3.30}{} & {$\infty$} &      \\
		\bottomrule
	\end{tabular}
    \label{tab:V3TabI}
\end{table}

\begin{table}[h]
    \centering
    \caption{Die 3-dB-Methode.}
    \begin{tabular}{SSSSSS}
        \toprule
		{$d_1$}     & {$d_2$}     & {1. Min}    & {2. Min}    & {$\lambda_g$} & {SWR}      \\
        {[mm]}      & {[mm]}      & {[mm]}      & {[mm]}      & {[mm]}        &            \\
        \midrule
		\SI{90.4}{} & \SI{88.4}{} & \SI{69.7}{} & \SI{94.2}{} & \SI{49}{}     & \SI{7.8}{} \\
		\bottomrule
	\end{tabular}
    \label{tab:V3TabII}
\end{table}

\begin{table}[h]
    \centering
    \caption{Die Abschwächer-Methode.}
    \begin{tabular}{SSSS}
        \toprule
		{$A_1$}   & {$A_2$}   & {$\Delta A$} & {SWR}     \\
        {[dB]}    & {[dB]}    & {[dB]}       &           \\
        \midrule
		\SI{20}{} & \SI{44}{} & \SI{24}{}    & \SI{12}{} \\
		\bottomrule
	\end{tabular}
    \label{tab:V3TabIII}
\end{table}