\section{Diskussion}
\label{sec:diskussion}

Im gesamten Versuch werden sehr wenig Messwerte aufgenommen; für diese dann lediglich ein einzelner Wert.
Demzufolge lässt sich nur sehr bedingt eine Aussage über die Genauigkeit der einzelnen Ergebnisse treffen.
Auffällig beim SWR-Meter ist aber, dass das Messgerät immer wieder stark um den Messwert oszilliert, weswegen das genaue Ablesen eines Wertes erheblich erschwert ist.

\subsection{Untersuchung eines Reflexklystrons}

Die Untersuchung dreier Moden des Klystrons mit einem Oszilloskop lässt sich ohne große Probleme durchführen.
Auch die Ergebnisse liegen in einem physikalisch sinnvollen Bereich.
Für höhere Reflektorspannungen ergeben sich im Allgemeinen auch höhere Leistungen der Mikrowellen.
Allerdings weichen die gemessenen Maximalspannungen von dem erwarteten Peak der Moden etwas ab, was die Ergebnisse leicht verfälscht.
Die Beobachtung des dips aufgrund der Einstellung des Frequenzmessers war an den Rändern einer Mode durch die Darstellung des Oszilloskopes oft nicht ganz leicht.

Bei der elektronischen Abstimmung wird eine Bandbreite von \SI{35}{\mega\hertz} für die Mode mit einer Mittenfrequenz von \SI{8997}{\mega\hertz} ermittelt und eine Abstimm-Empfindlichkeit des Klystrons von \SI{2.06}{\mega\hertz\per\volt}.
Die Reflektorspannungsabhängigkeit der Mikrowellenfrequenz ist sehr hoch, sodass diese sehr genau einstellbar sein muss, wenn Mikrowellen mit dem Klystron 2K25 zur Anwendung kommen sollen.

\subsection{Messung von Frequenz, Wellenlänge und Dämpfung}

Bei der direkten Frequenzmessung über den Frequenzmesser wird für die konfigurierte Mikrowelle eine Frequenz von \SI{9000}{\mega\hertz} gemessen.
Aus der Wellenlängenmessung einer stehenden Welle (und den Abmessungen des Hohlleiters) wird hingegen eine Frequenz von \SI{9129.1}{\mega\hertz} für diese Mikrowelle errechnet.
Die relative Abweichung beider Werte beträgt zwar nur \SI{1.4}{\percent}, ein Frequenzunterschied von \SI{129}{\mega\hertz} ist allerdings in manchen Anwendungen sicherlich ausreichend, um bestimmte Frequenzbereiche zu verlassen.
Für die direkte Messung ist hier nicht anzugeben, wie genau diese ist, da bei dem verwendeten Frequenzmesser nicht direkt zu überprüfen ist, wie genau die Anzeige ist.
Die Wellenlängen- und Hohlleitermessung hingegen ist mit bekannten Fehlern zu versehen, sodass die indirekte Frequenzmessung in diesem Versuch als die bessere anzusehen ist, die eingestellte Frequenz der Mikrowelle also eher ca. \SI{9132}{\mega\hertz} beträgt.

Die Dämpfungsmessung hat über den gesamten Bereich bis auf kleinere Abweichungen gut mit den aus der Theoriekurve abgelesenen Werten übereingestimmt.
In Abbildung~\ref{fig:daempfungskurve} liegen die Messpunkte anfangs und nach kurzem Ausscheren auch bei den größeren Dämpfungen gut übereinander.

\subsection{Stehwellen-Messungen}

Die mit den drei Methoden bestimmten SWR stimmen in der Größenordnung überein, wobei allerdings der Messwert bei $\SI{9}{\milli\meter}$ Sondentiefe einen klaren Ausreißer darstellt.
Eine Messung in den höheren Bereichen wurde auch hier durch die Fluktuationen im Ausschlag des Zeigers erschwert.
Unter den anderen Werten weicht der Kleinste, $\text{SWR}=\SI{1.16}{}$, um ca. $\SI{90}{\percent}$ vom größten Wert, $\text{SWR}=\SI{12}{}$, ab.
Bei der 3-dB-Methode liefert die Messung mit $\text{SWR}=\SI{7.8}{}$ einen mittig zwischen der SWR-Methode und der Abschwächer-Methode liegenden Wert.
Aus dieser großen Streuung lässt sich bereits auf eventuelle Messfehler und systematische Fehler schließen. 
Die Größenordnung lässt sich jedoch unter Ausschluss des unendlichen Wertes klar feststellen.
Die Verteilung entspricht allerdings der Erwartung, da die Abschwächermethode (auch hier unter Ausschluss der unendlich großen Welligkeit) wie vorgesehen die größte Welligkeit misst, während die direkte methode die kleinesten Werte liefert.
