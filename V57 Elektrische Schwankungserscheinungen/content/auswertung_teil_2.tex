\subsection{Untersuchung des Stromrauschens einer Oxydkathode}

Im Folgenden wird das Stromrauschen einer Oxydkathode genauer untersucht. Die
Kathode wird mit einem Kathodenstrom von~$I=\SI{1}{\milli\ampere}$ betrieben,
was außerhalb des Sättigungsbereichs liegt. Die Spannungen, die bei gegebener
Mittenfrequenz des Band- bzw. Selektivverstärkers am Arbeitswiderstand der
Kathode abfallen, sind in Tabelle~\ref{tab:oxydkathode} gezeigt.

Gemäß Formel~\ref{...} in Abschnitt~\ref{sec:theorie} berechnet sich die
Frequenzverteilung aus den mittleren Rauschspannungsquadraten zu
%
\begin{equation}
  W(\nu)=\frac{\bar{U}^2}{R^2\symup{\Delta}\nu}
\end{equation}
%
Die benötigten Bandbreiten~$\symup{\Delta}\nu$ bestimmen sich mit Hilfe der
Werte aus einer gegebenen empirisch ermittelten Tabelle, die dem Anhang
beigefügt ist. Abbildung~\ref{fig:oxydkathode_frequenzspektrum} zeigt die
Frequenzverteilung~$W$ aufgetragen gegen die Mittenfrequenz~$\nu$. Der in
Abschnitt~\ref{sec:theorie} beschriebene Funkel-Effekt ist besonders im
niederfrequenten Bereich des Spektrums dominant. Zur Bestimmung des
Exponenten~$\alpha$ in der Formel für den Funkel-Effekt
%
\begin{equation}
  W(\nu)\propto\frac{I_0^2}{\nu^{\alpha}}
\end{equation}
%
wird für den linearen Anteil der Frequenzverteilung eine Ausgleichsrechnung
durchgeführt. Abbildung~\ref{fig:oxydkathode_frequenzspektrum_linearer_teil}
zeigt den linearen Teil der Frequenzverteilung, sowie die Ausgleichsgerade. Der
Exponent~$\alpha$ wird damit zu
%
\begin{equation}
  \alpha=\num{1.092(28)}
\end{equation}
%
bestimmt.

\begin{figure}
  \includegraphics{analysis/oxydkathode_frequenzspektrum.pdf}
  \caption{Frequenzverteilung des Stromrauschens einer Oxydkathode. Zur besseren
  Übersichtlichkeit sind die Unsicherheiten der Messwerte nicht mit
  eingezeichnet.}
  \label{fig:oxydkathode_frequenzspektrum}
\end{figure}

\begin{figure}
  \includegraphics{analysis/oxydkathode_frequenzspektrum_linearer_teil.pdf}
  \caption{Linearer Anteil der Frequenzverteilung des Stromrauschens einer
  Oxydkathode. Aus der Steigung der Ausgleichsgeraden wird der Exponent~$\alpha$
  in der Formel für den Funkel-Effekt bestimmt.}
  \label{fig:oxydkathode_frequenzspektrum_linearer_teil}
\end{figure}

\subsection{Untersuchung des Stromrauschens einer Reinmetallkathode}
