\section{Durchführung}
\label{sec:durchführung}

\subsection{Untersuchung des Spannungsrauschen von ohmschen Widerständen}

Zunächst wird das Spannungsrauschen von zwei ohmschen Widerständen untersucht.
Dazu wird die in Abbildung~\ref{fig:aufbau_einfachschaltung} gezeigte Schaltung
(Einfachschaltung) aufgebaut. Tiefpass und Gleichspannungsverstärker sind in
einem Gerät verbaut. Der Bandfilter wird auf einen Frequenzbereich
von~\SI{4}{\kilo\hertz} bis~\SI{6}{\kilo\hertz} eingestellt. Die Vorverstärkung
beträgt x1000. Die Gleichspannungsverstärkung wird konstant auf x10 eingestellt.

\begin{figure}
  \centering
  \includegraphics[width=0.9\textwidth]{figures/aufbau_einfachschaltung.pdf}
  \caption{Schematischer Aufbau der Einfachschaltung zur Messung des thermischen
  Rauschens eines ohmschen Widerstandes~\cite{V57}.}
  \label{fig:aufbau_einfachschaltung}
\end{figure}

Um das Rauschen der Widerstände korrekt bestimmen zu können, muss das
Eigenrauschen der Einfachschaltung bekannt sein. Dazu wird der Widerstand in
Abbildung~\ref{fig:aufbau_einfachschaltung} durch einen Kurzschluss ersetzt. Bei
Variation der Nachverstärkung wird nun mit Hilfe des Voltmeters die Spannung
gemessen und protokolliert.

In einem nächsten Schritt wird eine Kalibrierungskurve für den Bandfilter
aufgenommen. Dazu wird der ohmsche Widerstand durch einen Sinus-Generator
ausgetauscht. Bei Variation der eingestellten Frequenz wird wieder die Spannung
mit dem Voltmeter gemessen. Die Nachverstärkung wird hier und in allen weiteren
Versuchsteilen stets so gewählt, dass die Overload-Lampe des Quadrierers gerade
nicht bzw. nur selten aufleuchtet.

Es werden zwei variable ohmsche Widerstände untersucht. Der erste Widerstand ist
von~\SI{50}{\ohm} bis~\SI{1000}{\ohm} kontinuerlich einstellbar. Der zweite
Widerstand geht von~\SI{1}{\kilo\ohm} bis~\SI{100}{\kilo\ohm}. Für jeden
eingestellten Widerstand wird die Spannung am Ende der Schaltung mit Hilfe des
Voltmeters gemessen. Die Widerstände selbst werden mit einem Ohmmeter bestimmt.

Um eine bessere Abschätzung für den Einfluss des Eigenrauschens zu erhalten,
wird eine weitere Messreihe mit einer abgewandelten Schaltung durchgeführt.
Abbildung~\ref{fig:aufbau_korrelatorschaltung} zeigt den Aufbau der sogenannten
Korrelatorschaltung. Der wesentliche Unterschied besteht darin, dass das Signal
vom Widerstand ausgehend aufgespalten und durch verschiedene Verstärker und
Filter zum Quadrierer gelangt. Der Bandpass wurde zudem durch zwei
Selektivverstärker mit der Einstellung~\SI{5}{\kilo\hertz} ersetzt.

\begin{figure}
  \centering
  \includegraphics[width=0.9\textwidth]{figures/aufbau_korrelatorschaltung.pdf}
  \caption{Schematischer Aufbau der Korrelatorschaltung zur Messung des
  thermischen Rauschens eines ohmschen Widerstandes~\cite{V57}.}
  \label{fig:aufbau_korrelatorschaltung}
\end{figure}

Auch für diese Schaltung wird zunächst wie oben beschrieben eine
Kalibrierungskurve aufgenommen. Danach erfolgt ebenso analog eine Messreihe mit
den Widerständen.

\subsection{Untersuchung des Stromrauschens verschiedener Kathoden}

% Reinmetallkathode
