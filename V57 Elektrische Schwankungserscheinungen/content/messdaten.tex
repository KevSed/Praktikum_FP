\section{Messdaten}
\label{sec:Messdaten}

\begin{table}
  \centering
  \begin{tabular}{S[table-format=1.3]
                  S[table-format=3.0]
                  S[table-format=1.2]
                  S[table-format=1.2(2)]
                  S[table-format=1.4]}
    \toprule
    {$\nu/\si{\kilo\hertz}$} & {$V_{\symup{N}}$} & {$U_{\symup{gemessen}}/\si{\volt}$} & {$U_{\symup{einheitlich}}/\si{\volt}$} & {Durchlasskoeffizient $\beta$} \\
    \midrule
    1.078 & 500 & 0.51(1)  & 0.204(2)  &  0.0034  \\
    1.586 & 200 & 0.09(1)  & 0.0223(2) &  0.0036  \\
    2.036 & 200 & 0.31(1)  & 0.0790(8) &  0.0132  \\
    2.530 & 200 & 1.44(1)  & 0.360(4)  &  0.0613  \\
    3.021 & 200 & 5.06(1)  & 1.27(1)   &  0.2158  \\
    3.267 & 100 & 1.99(1)  & 1.99(2)   &  0.4878  \\
    3.485 & 100 & 2.86(1)  & 2.86(3)   &  0.8173  \\
    3.725 & 100 & 3.89(1)  & 3.89(4)   &  1.0000  \\
    3.955 & 100 & 4.79(1)  & 4.79(5)   &  0.9351  \\
    4.205 & 100 & 5.51(1)  & 5.51(6)   &  0.7371  \\
    4.444 & 100 & 5.86(1)  & 5.86(6)   &  0.5066  \\
    4.773 & 100 & 5.78(1)  & 5.78(6)   &  0.3164  \\
    4.963 & 100 & 5.48(1)  & 5.48(5)   &  0.1991  \\
    5.263 & 100 & 4.77(1)  & 4.77(5)   &  0.1249  \\
    5.424 & 100 & 4.32(1)  & 4.32(4)   &  0.0755  \\
    5.793 & 100 & 3.29(1)  & 3.29(3)   &  0.0537  \\
    5.915 & 100 & 2.97(1)  & 2.97(3)   &  0.0353  \\
    6.432 & 100 & 1.86(1)  & 1.860(2)  &  0.0237  \\
    6.916 & 100 & 1.17(1)  & 1.170(1)  &  0.0166  \\
    7.445 & 200 & 2.93(1)  & 0.733(7)  &  0.3393  \\
    8.016 & 200 & 1.77(1)  & 0.443(4)  &  0.6637  \\
    8.412 & 200 & 1.26(1)  & 0.316(3)  &  0.9402  \\
    8.895 & 200 & 0.83(1)  & 0.208(2)  &  0.9863  \\
    9.432 & 200 & 0.56(1)  & 0.140(1)  &  0.8139  \\
    9.936 & 200 & 0.39(1)  & 0.098(1)  &  0.5612  \\
    \bottomrule
  \end{tabular}
\caption{Messdaten der Kalibrationsmessung der Einfachschaltung bei einer
konstanten Vorverstärkung von \num{1000} und Gleichspannungsverstärkung von
\num{10}, sowie die oben abgebildeten Werte für den Durchlasskoeffizienten. Die Messunsicherheit der Spannungen wird auf $\SI{1}{\percent}$ geschätzt. Die Spannungen sind auf eine Nachverstärkung von \num{100}
vereinheilicht und haben den entsprechenden Fehler angegeben.}
  \label{tab:1fach_kalib}
\end{table}


\begin{table}
  \centering
  \begin{tabular}{S[table-format=4.0]
                  S[table-format=2.1]}
    \toprule
    {$V_{\symup{N}}$} & {$U_{\symup{R}}/\si{\milli\volt}$} \\
    \midrule
    1    &  3.1 \\
    2    &  3.0 \\
    5    &  3.1 \\
    10   &  3.1 \\
    20   &  3.0 \\
    50   &  3.0 \\
    100  &  3.0 \\
    200  &  3.9 \\
    500  & 11.6 \\
    1000 & 36.2 \\
    \bottomrule
  \end{tabular}
\caption{Messdaten zur Bestimmung des Eigenrauschens des verwendeten
Verstärkers. Gemessen bei einer Vorverstärkung von $V_V=1000$ und einer
Gleichspannungsverstärkung von $V_==10$.}
  \label{tab:eigenrauschen}
\end{table}

\begin{table}
  \centering
  \begin{tabular}{S[table-format=3.0]
                  S[table-format=4.0]
                  S[table-format=1.3]
                  S[table-format=2.2]}
    \toprule
    {$R/\si{\ohm}$} & {$V_{\symup{N}}$} & {$U_{\symup{R}}/\si{\volt}$} & {$U_\text{korr}^2/\si{\volt^2}\cdot10^{-15}$} \\
    \midrule
    50  & 1000 & 0.049 & 1.28 \\
    102 & 1000 & 0.064 & 2.78 \\
    152 & 1000 & 0.078 & 4.18 \\
    207 & 1000 & 0.093 & 5.68 \\
    251 & 1000 & 0.104 & 6.78 \\
    303 & 1000 & 0.114 & 7.78 \\
    353 & 1000 & 0.129 & 9.28 \\
    404 & 1000 & 0.144 & 10.78 \\
    451 & 1000 & 0.152 & 11.58 \\
    501 & 1000 & 0.167 & 13.08 \\
    555 & 1000 & 0.185 & 14.88 \\
    605 & 1000 & 0.194 & 15.78 \\
    652 & 1000 & 0.208 & 17.18 \\
    704 & 1000 & 0.222 & 18.58 \\
    752 & 1000 & 0.237 & 20.08 \\
    807 & 1000 & 0.250 & 21.38 \\
    855 & 1000 & 0.261 & 22.48 \\
    908 & 1000 & 0.274 & 23.78 \\
    949 & 1000 & 0.283 & 24.68 \\
    995 & 1000 & 0.296 & 25.98 \\
    \bottomrule
  \end{tabular}
  \caption{Messdaten, sowie die korrigierten und oben abgebildeten Werte der Einfachschaltung zur Bestimmung des thermischen Rauschens des schwachen Widerstandes. Gemessen bei einer Vorverstärkung von $V_V=1000$ und einer Gleichspannungsverstärkung von $V_==10$. Die Fehler der gemessenen Spannungen sind auf $\SI{1}{\percent}$ geschätzt.}
  \label{tab:1fach_schwach}
\end{table}

\begin{table}
  \centering
  \begin{tabular}{S[table-format=5.0]
                  S[table-format=4.0]
                  S[table-format=1.3]
                  S[table-format=2.3]}
    \toprule
    {$R/\si{\ohm}$} & {$V_{\symup{N}}$} & {$U_{\symup{R}}/\si{\volt}$} & {$U_\text{korr}^2/\si{\volt^2}\cdot10^{-13}$} \\
    \midrule
    1330  & 1000 & 0.384 & 0.384 \\
    5320  & 1000 & 1.378 & 1.378 \\
    10240 &  500 & 0.704 & 2.816 \\
    15030 &  500 & 1.003 & 4.012 \\
    19600 &  500 & 1.396 & 5.584 \\
    25100 &  500 & 1.691 & 6.764 \\
    30800 &  200 & 0.337 & 8.425 \\
    35100 &  100 & 0.173 & 17.300 \\
    40600 &  200 & 0.472 & 11.800 \\
    44700 &  200 & 0.492 & 12.300 \\
    50700 &  200 & 0.547 & 13.675 \\
    56300 &  200 & 0.714 & 17.850 \\
    60000 &  100 & 0.195 & 19.500 \\
    64800 &  200 & 0.731 & 18.275 \\
    70100 &  100 & 0.213 & 21.300 \\
    76600 &  100 & 0.262 & 26.200 \\
    80200 &  100 & 0.245 & 24.500 \\
    85700 &  100 & 0.262 & 26.200 \\
    91700 &  100 & 0.301 & 30.100 \\
    95200 &  100 & 0.336 & 33.600 \\
    97300 &  100 & 0.367 & 36.700 \\
    \bottomrule
  \end{tabular}
  \caption{Messdaten, sowie die korrigierten und oben abgebildeten Werte der Einfachschaltung zur Bestimmung des thermischen Rauschens des starken Widerstandes. Gemessen bei einer Vorverstärkung von $V_V=1000$ und einer Gleichspannungsverstärkung von $V_==10$. Die Fehler der gemessenen Spannungen sind auf $\SI{1}{\percent}$ geschätzt.}
  \label{tab:1fach_stark}
\end{table}

\begin{table}
  \centering
  \begin{tabular}{S[table-format=1.3]
                  S[table-format=2.0]
                  S[table-format=1.2]
                  S[table-format=1.4]}
    \toprule
    {$f/\si{\kilo\hertz}$} & {$V_{\symup{N}}$} & {$U/\si{\volt}$} & {Durchlasskoeffizient $\beta$} \\
    \midrule
    1.088 & 50 & 0.23 & 0.0011 \\
    1.609 & 50 & 0.42 & 0.0019 \\
    2.071 & 50 & 0.70 & 0.0032 \\
    2.564 & 50 & 1.27 & 0.0059 \\
    3.034 & 50 & 2.27 & 0.0105 \\
    3.523 & 50 & 4.60 & 0.0213 \\
    4.033 & 20 & 1.88 & 0.0545 \\
    4.480 & 20 & 5.81 & 0.1683 \\
    4.943 & 10 & 7.17 & 0.8308 \\
    5.484 & 10 & 2.65 & 0.3071 \\
    5.976 & 20 & 3.49 & 0.1011 \\
    6.424 & 20 & 1.72 & 0.0498 \\
    6.920 & 50 & 6.20 & 0.0287 \\
    7.410 & 20 & 0.70 & 0.0202 \\
    7.926 & 50 & 2.90 & 0.0134 \\
    8.481 & 50 & 2.29 & 0.0106 \\
    9.064 & 50 & 1.75 & 0.0081 \\
    9.559 & 50 & 1.50 & 0.0070 \\
    9.940 & 50 & 1.32 & 0.0061 \\
    4.211 & 20 & 2.80 & 0.0811 \\
    4.750 & 10 & 3.54 & 0.4102 \\
    5.246 & 10 & 6.00 & 0.6952 \\
    5.755 & 20 & 5.39 & 0.1561 \\
    5.072 & 10 & 8.63 & 1.0000 \\
    5.182 & 10 & 7.27 & 0.8424 \\
    \bottomrule
  \end{tabular}
  \caption{Messdaten zur Kalibrationsmessung der Korrelatorschaltung, sowie der oben abgebildete Durchlasskoeffizient $\beta$. Gemessen über eine sinus-Spannung variabler Frequenz $f$ bei einer Vorverstärkung von $V_V=1000$ und einer Gleichspannungsverstärkung von $V_==10$. Die Fehler der gemessenen Spannungen sind auf $\SI{1}{\percent}$ geschätzt.}
  \label{tab:kalib_korr}
\end{table}


\begin{table}
  \centering
  \begin{tabular}{S[table-format=3.0]
                  S[table-format=4.0]
                  S[table-format=1.3]
                  S[table-format=2.3]}
    \toprule
    {$R/\si{\ohm}$} & {$V_{\symup{N}}$} & {$U_{\symup{R}}/\si{\volt}$} & {$U_\text{korr}^2/\si{\volt^2}\cdot10^{-15}$} \\
    \midrule
    50  & 1000 & 0.049 & 1.500 \\
    102 & 1000 & 0.064 & 2.250 \\
    152 & 1000 & 0.078 & 1.700 \\
    207 & 1000 & 0.093 & 2.248 \\
    251 & 1000 & 0.104 & 2.728 \\
    303 & 1000 & 0.114 & 3.300 \\
    353 & 1000 & 0.129 & 3.820 \\
    404 & 1000 & 0.144 & 4.475 \\
    451 & 1000 & 0.152 & 5.150 \\
    501 & 1000 & 0.167 & 5.625 \\
    555 & 1000 & 0.185 & 6.275 \\
    605 & 1000 & 0.194 & 6.750 \\
    652 & 1000 & 0.208 & 7.425 \\
    704 & 1000 & 0.222 & 7.825 \\
    752 & 1000 & 0.237 & 8.250 \\
    807 & 1000 & 0.250 & 9.100 \\
    855 & 1000 & 0.261 & 9.625 \\
    908 & 1000 & 0.274 & 10.200 \\
    949 & 1000 & 0.283 & 10.625 \\
    995 & 1000 & 0.296 & 11.250 \\
    \bottomrule
  \end{tabular}
  \caption{Messdaten, sowie die korrigierten und oben abgebildeten Werte der Korrelatorschaltung zur Bestimmung des thermischen Rauschens des schwachen Widerstandes. Gemessen bei einer Vorverstärkung von $V_V=1000$ und einer Gleichspannungsverstärkung von $V_==10$. Die Fehler der gemessenen Spannungen sind auf $\SI{1}{\percent}$ geschätzt.}
  \label{tab:schwach_korr}
\end{table}

\begin{table}
  \centering
  \begin{tabular}{S[table-format=5.0]
                  S[table-format=4.0]
                  S[table-format=1.3]
                  S[table-format=2.3]}
    \toprule
    {$R/\si{\ohm}$} & {$V_{\symup{N}}$} & {$U_{\symup{R}}/\si{\volt}$} & {$U_\text{korr}^2/\si{\volt^2}\cdot10^{-13}$} \\
    \midrule
    5070 & 1000 & 0.384 & 0.543 \\
    10130 & 1000 & 1.378 & 1.073 \\
    15120 &  500 & 0.704 & 1.594 \\
    20000 &  500 & 1.003 & 2.156 \\
    25000 &  500 & 1.396 & 2.884 \\
    30000 &  500 & 1.691 & 3.588 \\
    40100 &  200 & 0.337 & 5.250 \\
    44900 &  100 & 0.173 & 5.800 \\
    50300 &  200 & 0.472 & 6.750 \\
    54900 &  200 & 0.492 & 7.375 \\
    60300 &  200 & 0.547 & 8.225 \\
    64900 &  200 & 0.714 & 8.750 \\
    70200 &  100 & 0.195 & 9.475 \\
    74900 &  200 & 0.731 & 10.125 \\
    80300 &  100 & 0.213 & 11.025 \\
    85100 &  100 & 0.262 & 11.875 \\
    90100 &  100 & 0.245 & 13.300 \\
    95200 &  100 & 0.262 & 14.100 \\
    \bottomrule
  \end{tabular}
  \caption{Messdaten, sowie die korrigierten und oben abgebildeten Werte der Korrelatorschaltung zur Bestimmung des thermischen Rauschens des starken Widerstandes. Gemessen bei einer Vorverstärkung von $V_V=1000$ und einer Gleichspannungsverstärkung von $V_==10$. Die Fehler der gemessenen Spannungen sind auf $\SI{1}{\percent}$ geschätzt.}
  \label{tab:stark_korr}
\end{table}

\begin{table}
  \centering
  \begin{tabular}{S[table-format=6.0]
                  S[table-format=2.0]
                  S[table-format=2.0]
                  S[table-format=1.2(2)]
                  S[table-format=2.4(4)]
                  S[table-format=5.2(3)]
                  S[table-format=3.3(3)]}
    \toprule
    {$\nu$/\si{\hertz}} & {$V_{\symup{S}}$} & {$V_{\symup{N}}$} &
    {$U_{\symup{gemessen}}/\si{\volt}$} &
    {$U_{\symup{einheitlich}}/\si{\volt}$} & {$\symup{\Delta}\nu/\si{\hertz}$} &
    {$W(\nu)\cdot10^{21}/\si{\ampere\squared\second}$} \\
    \midrule
    460000 &  1 & 50 & 1.55(1) & 24.80(16)   & 34300(200) &   0.373(3) \\
    400000 &  1 & 50 & 1.71(1) & 27.36(16)   & 35800(200) &   0.395(3) \\
    360000 &  1 & 50 & 1.70(1) & 27.20(16)   & 35200(300) &   0.399(4) \\
    340000 &  1 & 50 & 1.67(1) & 26.72(16)   & 33500(500) &   0.412(7) \\
    320000 &  1 & 50 & 1.68(1) & 26.88(16)   & 33000(500) &   0.421(7) \\
    280000 &  1 & 50 & 1.35(1) & 21.60(16)   & 27500(500) &   0.406(8) \\
    240000 &  1 & 50 & 1.20(1) & 19.20(16)   & 24800(400) &   0.400(7) \\
    200000 &  1 & 50 & 1.01(1) & 16.16(16)   & 21500(200) &   0.388(5) \\
    160000 &  1 & 50 & 0.78(1) & 12.48(16)   & 16300      &   0.395(5) \\
    120000 &  1 & 50 & 0.62(1) &  9.92(16)   & 12200      &   0.420(7) \\
    100000 & 10 &  5 & 0.68(1) & 10.88(16)   & 12550      &   0.448(7) \\
     80000 & 10 &  5 & 0.56(1) &  8.96(16)   & 10250      &   0.451(8) \\
     60000 & 10 & 10 & 1.70(1) &  6.80(04)   &  7950      &   0.442(2) \\
     40000 & 10 & 10 & 1.20(1) &  4.800(4)   &  5450      &   0.455(4) \\
     30000 & 10 &  5 & 0.25(1) &  4.00(16)   &  4100      &   0.50(2)  \\
     20000 & 10 & 10 & 0.67(1) &  2.680(4)   &  2750      &   0.503(8) \\
     10000 & 10 &  5 & 0.14(1) &  2.24(16)   &  1400.7    &   0.83(6)  \\
      9000 & 10 &  5 & 0.10(1) &  1.60(16)   &  1260.7    &   0.66(6)  \\
      7000 & 10 & 10 & 0.30(1) &  1.200(4)   &   980.7    &   0.63(2)  \\
      5000 & 10 & 20 & 1.00(1) &  1.000(1)   &   700.7    &   0.737(7) \\
      3000 & 10 & 20 & 0.70(1) &  0.700(1)   &   420.7    &   0.86(1)  \\
      2000 & 10 & 20 & 0.56(1) &  0.560(1)   &   280.7    &   1.03(2)  \\
      1000 & 10 & 20 & 0.41(1) &  0.410(1)   &   140.7    &   1.50(4)  \\
       900 & 10 & 20 & 0.36(1) &  0.360(1)   &   126.7    &   1.47(4)  \\
       700 & 10 & 20 & 0.34(1) &  0.340(1)   &    98.7    &   1.78(5)  \\
       500 & 10 & 10 & 0.08(1) &  0.320(4)   &    70.7    &   2.3(3)   \\
       300 & 10 & 20 & 0.32(1) &  0.320(1)   &    42.7    &   3.9(1)   \\
       200 & 10 & 20 & 0.30(1) &  0.300(1)   &    28.7    &   5.4(2)   \\
       100 & 10 & 20 & 0.35(1) &  0.350(1)   &    14.7    &  12.3(4)   \\
        90 & 10 & 20 & 0.40(1) &  0.400(1)   &    13.2    &  15.6(4)   \\
        70 & 10 & 20 & 0.45(1) &  0.450(1)   &    10.2    &  22.8(5)   \\
        30 & 10 & 20 & 0.52(1) &  0.520(1)   &     4.2    &  64(1)     \\
        20 & 10 & 20 & 0.58(1) &  0.580(1)   &     2.7    & 111(2)     \\
        10 & 10 & 20 & 0.65(1) &  0.650(1)   &     1.2    & 280(4)     \\
         9 & 10 & 50 & 2.50(1) &  0.4000(16) &     1.05   & 196.8(8)   \\
         7 & 10 & 50 & 2.20(1) &  0.3520(16) &     0.75   & 242(1)     \\
         5 & 10 & 50 & 2.30(1) &  0.3680(16) &     0.45   & 422(2)     \\
         3 & 10 &100 & 3.40(1) &  0.1360(4)  &     0.15   & 468(1)     \\
    \bottomrule
  \end{tabular}
  \caption{Messdaten zur Untersuchung einer Oxydkathode, bei einer  konstanten
  Vorverstärkung von 1000 und Gleichspannungsverstärkung von 10. Die Spannungen
  sind auf eine Nachverstärkung von 20 vereinheilicht.}
  \label{tab:oxydkathode}
\end{table}

\begin{table}
  \centering
  \begin{tabular}{S[table-format=6.0]
                  S[table-format=2.0]
                  S[table-format=3.0]
                  S[table-format=1.2(2)]
                  S[table-format=3.4(4)]
                  S[table-format=5.2(3)]
                  S[table-format=3.3(3)]}
    \toprule
    {$\nu/\si{\hertz}$} & {$V_{\symup{S}}$} & {$V_{\symup{N}}$} &
    {$U_{\symup{gemessen}}/\si{\volt}$} &
    {$U_{\symup{einheitlich}}/\si{\volt}$} & {$\symup{\Delta}\nu/\si{\hertz}$} &
    {$W(\nu)\cdot10^{21}/\si{\ampere\squared\second}$} \\
    \midrule
    460000 &  1 &  50 & 2.98(1) & 298(1)        & 21500(400) &   0.253(5) \\
    400000 &  1 &  50 & 3.22(1) & 322(1)        & 24300(400) &   0.242(4) \\
    360000 &  1 &  20 & 0.57(1) & 356.25(625)   & 24400(400) &   0.267(6) \\
    320000 &  1 &  20 & 0.54(1) & 337.50(625)   & 24000(400) &   0.257(6) \\
    280000 &  1 &  20 & 0.52(1) & 325.00(625)   & 21300(200) &   0.279(6) \\
    240000 &  1 &  50 & 2.93(1) & 293(1)        & 20900(200) &   0.256(3) \\
    200000 &  1 &  50 & 2.77(1) & 277(1)        & 18800      &   0.269(1) \\
    160000 &  1 &  50 & 2.10(1) & 210(1)        & 14400      &   0.266(1) \\
    120000 &  1 &  50 & 1.75(1) & 175(1)        & 11600      &   0.276(2) \\
    100000 & 10 &   5 & 2.10(1) & 210(1)        & 12100      &   0.317(2) \\
     90000 & 10 &   5 & 1.94(1) & 194(1)        & 11010      &   0.322(2) \\
     70000 & 10 &   5 & 1.57(1) & 157(1)        &  8830      &   0.325(2) \\
     50000 & 10 &   5 & 1.18(1) & 118(1)        &  6800      &   0.317(3) \\
     30000 & 10 &   5 & 0.75(1) &  75(1)        &  4100      &   0.334(4) \\
     20000 & 10 &   5 & 0.51(1) &  51(1)        &  2750      &   0.339(7) \\
     10000 & 10 &  10 & 1.02(1) &  25.50(25)    &  1400.7    &   0.332(3) \\
      9000 & 10 &  10 & 0.88(1) &  22.00(25)    &  1260.7    &   0.319(4) \\
      7000 & 10 &  10 & 0.68(1) &  17.00(25)    &   980.7    &   0.316(5) \\
      5000 & 10 &  10 & 0.49(1) &  12.25(25)    &   700.7    &   0.319(6) \\
      3000 & 10 &  20 & 1.28(1) &   8.0000(625) &   420.7    &   0.347(3) \\
      2000 & 10 &  20 & 0.87(1) &   5.4375(625) &   280.7    &   0.354(4) \\
      1000 & 10 &  20 & 0.42(1) &   2.6250(625) &   140.7    &   0.341(8) \\
       900 & 10 &  50 & 2.02(1) &   2.02(1)     &   126.7    &   0.291(1) \\
       700 & 10 &  50 & 1.65(1) &   1.65(1)     &    98.7    &   0.305(2) \\
       500 & 10 &  50 & 1.23(1) &   1.23(1)     &    70.7    &   0.318(2) \\
       300 & 10 &  50 & 0.68(1) &   0.68(1)     &    42.7    &   0.291(4) \\
       200 & 10 & 100 & 1.98(1) &   0.4950(25)  &    28.7    &   0.315(2) \\
       100 & 10 & 100 & 0.98(1) &   0.2450(25)  &    14.7    &   0.304(3) \\
        90 & 10 & 100 & 1.08(1) &   0.2700(25)  &    13.2    &   0.374(3) \\
        70 & 10 & 100 & 0.95(1) &   0.2375(25)  &    10.2    &   0.425(4) \\
        30 & 10 & 100 & 1.20(1) &   0.3000(25)  &     4.2    &   1.30(1)  \\
        20 & 10 &  50 & 0.80(1) &   0.80(1)     &     2.7    &   5.41(7)  \\
        10 & 10 &  20 & 0.65(1) &   4.0625(625) &     1.2    &  62(1)     \\
         5 & 10 &  20 & 1.30(1) &   8.1250(625) &     0.45   & 330(2)     \\
    \bottomrule
  \end{tabular}
  \caption{Messdaten zur Untersuchung einer Reinmetallkathode, bei einer
  konstanten Vorverstärkung von 1000 und Gleichspannungsverstärkung von 10. Die
  Spannungen sind auf eine Nachverstärkung von 50 vereinheilicht.}
  \label{tab:reinmetallkathode}
\end{table}
