\section{Diskussion}
\label{sec:diskussion}
%
Die im Rahmen dieses Experimentes ermittelten Größen stellen gut zu der Theorie
passende Werte dar. Allerdings haben sich einige experimentelle Umstände in Form
von Abweichungen in den Messwerten niedergeschlagen. \\
Die Vermessung der optischen Stabilität des Lasers ist mit dem vorliegenden
Versuchsaufbau nicht gut durchführbar. Einerseits ist die berechnete maximale
Länge des Resonators aufgrund der Kürze der optischen Bank nicht erreichbar und
somit auch nicht vermessbar. Außerdem ist eine stückweise Vergrößerung der
Resonatorlänge prinzipiell schwierig, da die Bauteile eine enorme Instabilität
der optischen Achse verursachen. Bereits kleinste Abweichungen in der
Ausrichtung der Spiegel und Sammellinsen sorgen für ein Abbrechen der Laserfunktion,
was das Aufnehmen einer Messreihe sehr verzögert und fehlerbehaftet. Hier wäre
die Verwendung einer möglichst geraden und stabilen Arbeitsfläche hilfreich. \\
Alle Messungen unterliegen dieser Instabilität geschuldet auch systematischen
Schwankungen in der Messung der Intensität. Diese variiert dahingehend, dass
die Leistung des Lasers abhängig von der Genauigkeit der Ausrichtung der Spiegel
untereinander, sowie der Genauigkeit des Treffpunktes von Laserstrahl und
Photodiode ist. Daher sorgen auch hier Variationen der Resonatorlänge stets für
systematische Fehler. \\
Die Vermessung der TEM-Moden ist mit dem Versuchsaufbau gut umsetzbar.
Allerdings fällt bei der~$\text{TEM}_{10}$-Mode auf, dass die beiden lokalen
Maxima unterschiedlich hoch sind. Dies kann zum Beispiel an der Asymmetrie
des verwendeten Drahtes liegen. Dieser hat eine endliche Ausdehnung und wirft somit
gegebenenfalls einen asymmetrichen Schatten. \\
Die Berechnung der Wellenlänge des Lasers liegt mit einem Mittelwert von $\SI{638(4)}{\nano\meter}$ nah am
tatsächlichen Wert von $\SI{633}{\nano\meter}$ \cite{laser} in der richtigen Größenordnung.
Die absolute Abweichung von~$\SI{5}{\nano\meter}$ ist für die meisten Messungen
allerdings deutlich zu groß. Es fällt hier auch auf, dass die bestimmten Wellenlängen
mit steigender Ordnung tendenziell auch größer werden. Dies lässt sich vermutlich
auf die Kleinwinkelnäherung zurückführen, die bei größer werdenden Winkeln in ihrer
Ungenauigkeit zunimmmt. \\
Auch die Polarisation des Laserlichtes entspricht der erwarteten Verteilung sehr
gut. Zusammenfassend lässt sich der Versuch als sehr gut geeignet zur
Untersuchung der Charakteristika eines Lasers bezeichnen. Die Funktionsfähigkeit
und die Genauigkeit der Untersuchungen genügen den allgemein gehaltenen
Ansprüchen dieses Versuches.
