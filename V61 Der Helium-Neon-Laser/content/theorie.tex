\section{Theorie}
\label{sec:theorie}

Laser (Light Amplification by Stimulated Emission of Radiation) bestehen stets
aus drei Hauptkomponenten: Pumpquelle, Resonator und aktives Lasermedium. Im
Folgenden soll die Funktionsweise eines Laser sowie die Eigenschaften der
Laserstrahlung kurz erläutert werden:

Das Strahlungsspektrum eines Lasers wird durch die Eigenschaften des aktiven
Lasermediums festgelegt. Ziel ist es, das Lasermedium so zu beeinflussen, dass
durch Wechselwirkung mit dem Strahlungsfeld eine Verstärkung der eingehenden
Strahlung erfolgt. Wesentlich für den Verstärkungsprozess ist die im Folgenden
diskutierte, durch einfallende Photonen ausgelöste induzierte Emission. Zur
Vereinfachung wird ein idealisiertes 2-Niveau-System eines Atoms im Lasermedium
betrachtet. Gemäß der Maxwell-Boltzmann-Verteilung überwiegt im thermischen
Gleichgewicht die Besetzung des Grundzustandes. Durch äußere Anregung, zum
Beispiel in Form von elektrischer Entladung, werden Elektronen auf ein höheres
Energieniveau gebracht. Tritt nun ein einfallendes Photon in Wechselwirkung mit
dem angeregten Atom, so kann es zu einer induzierten Emission kommen. Dabei wir
ein weiteres Photon derselben Energie, Phase und Ausbreitungsrichtung abgegeben.
