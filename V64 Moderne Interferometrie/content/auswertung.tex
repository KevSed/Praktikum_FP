\section{Auswertung}
\label{sec:auswertung}

Die in der Auswertung verwendeten Mittelwerte mehrfach gemessener Größen sind
gemäß der Gleichung
%
\begin{equation}
    \bar{x}=\frac{1}{n}\sum_{i=1}^n x_i
    \label{eq:mittelwert}
\end{equation}
%
bestimmt. Die Standardabweichung des Mittelwertes ergibt sich dabei zu
%
\begin{equation}
    \symup{\Delta}\bar{x}=\sqrt{\frac{1}{n(n-1)}\sum_{i=1}^n\left(x_i-\bar{x}\right)^2}.
    \label{eq:standardabweichung}
\end{equation}
%
Resultiert eine Größe über eine Gleichung aus zwei anderen fehlerbehafteten
Größen, so berechnet sich der Gesamtfehler nach der Gaußschen
Fehlerfortpflanzung zu
%
\begin{equation}
    \symup{\Delta}f(x_1,x_2,...,x_n)=\sqrt{\left(\frac{\partial f}{\partial x_1}\symup{\Delta}x_1\right)^2+\left(\frac{\partial f}{\partial x_2}\symup{\Delta}x_2\right)^2+ \dotsb +\left(\frac{\partial f}{\partial x_n}\symup{\Delta}x_n\right)^2}.
    \label{eq:fehlerfortpflanzung}
\end{equation}
%
Alle in der Auswertung angegebenen Größen sind stets auf die erste signifikante
Stelle des Fehlers gerundet. Setzt sich eine Größe über mehrere Schritte aus
anderen Größen zusammen, so wird erst am Ende gerundet, um Fehler zu vermeiden.
Zur Auswertung wird die Programmiersprache \texttt{python (Version 3.5.1)} mit
den Bibliothekserweiterungen \texttt{numpy}~\cite{numpy},
\texttt{scipy}~\cite{scipy} und \texttt{matplotlib}~\cite{matplotlib} verwendet.


\subsection{Kontrastbestimmung}
%
Zu Beginn wird das Kontrastmaximum des Sagnac-Interferometers gesucht; die
Einstellung, bei welcher das erzeugte Interferenzbild seine maximale Schärfe
besitzt. Dazu werden die maximale und die minimale Stromstärke an der
Photodiode für verschiedene Winkel des Polarisationsfilters über Variation der
Ausrichtung des Glasplättchens gemessen. Der Kontrast $K(\varphi)$ lässt sich
über den Zusammenhang~\eqref{eq:kontrast} berechnen. Er folgt in Abhängigkeit
des Winkels $\varphi$ einem Verlauf der etwa
%
\begin{equation}
  K(\varphi)=A\cdot\sin(2\varphi)^2
  \label{eq:kontrast_theo}
\end{equation}
%
entspricht. Die Messdaten und der angepasste Verlauf der
Funktion~\eqref{eq:kontrast_theo} sind in Abbildung~\ref{fig:kontrast}
dargestellt. Es folgen für die Maxima des Kontrastes die Winkel
$\varphi_1=\SI{45}{\degree}$ und $\varphi_2=\SI{135}{\degree}$.

\begin{figure}[htb]
  \centering
  \includegraphics[width=0.8\textwidth]{analysis/kontrast.pdf}
  \caption{Messwerte für den winkelabhängigen Verlauf des Kontrastes des verwendeten Interferometers, sowie die in \eqref{eq:kontrast_theo} beschriebene Funktion.}
  \label{fig:kontrast}
\end{figure}

Das Interferometer wird daher im Weiteren so eingestellt, dass bei
maximalem Kontrast gemessen wird.

\subsection{Bestimmung des Brechungsindex eines Glasplättchens}
%
Der Brechungsindex der Glasplättchen wird über die Zählung der Interferenzmaxima für verschiedene Winkel der Plättchen zum einfallenden Lichtstrahl bestimmt. Dazu werden drei unabhängige Messungen in Schritten von $\SI{10}{\degree}$ durchgeführt. Die jeweils bestimmten Brechungsindices ergeben schließlich den in~\eqref{eq:glas} aufgeführten Wert.
%
\begin{equation}
  n_\text{glas}= 1,24\,\pm\,0,26
  \label{eq:glas}
\end{equation}
%
Die hier verwendeten Glasplättchen zeigen eine Dicke von $d=\SI{0.5}{\milli\meter}$ und der Laser eine Wellenlänge von $\lambda=\SI{632.99}{\nano\meter}$. Der Zusammenhang für den Brechungsindex folgt damit aus den geometrischen Überlegungen zu
%
\begin{equation}
  n = \frac{1}{1 - \frac{N \cdot \lambda}{(d \cdot (\varphi_2^2 - \varphi_1^2))}}
\end{equation}

\subsection{Bestimmung des Brechungsindex von Kohlenstoffdioxid und Luft}
%
Nun wird das Lichtbrechungsverhalten zweier Gase in Abhängigkeit des Gasdrucks
untersucht. Dazu werden das Gas Kohlenstoffdioxid sowie das Gasgemisch Luft
gewählt. In einer Lichtdurchlässigen Kammer lässt sich der Gasdruck mit Hilfe
einer Pumpe regulieren, sodass der Phasenunterschied reguliert und somit der
Brechungsindex der Gase über eine Zählung von Interferenzmaxima bestimmt werden
kann. Die Länge der in diesem Experiment verwendeten Kammer beträgt
$L=\SI{100.0(1)}{\milli\meter}$~\cite{V64}. Die Messung wird über einen
Druckbereich von etwa $\SI{50}{\milli\bar}$ bis $\SI{1000}{\milli\bar}$ jeweils
dreimal durchgeführt. In den Abbildungen~\ref{fig:luft1} - \ref{fig:co2_3} sind die bestimmten
Brechungsindices in Abhängigkeit des Gasdrucks, sowie linearen Ausgleichsgeraden der in \eqref{eq:regr} dargestellten Form aufgetragen.
%
\begin{equation}
  n^2(p)=a\cdot p + b
  \label{eq:regr}
\end{equation}
%
Die Ergebnisse der Ausgleichsrechnungen sind in Tabelle~\ref{tab:lin} aufgetragen.

Aus diesen Messreihen ergeben sich für die linearen Zusammenhänge
folgende Mittelwerte:
%
\begin{align*}
  n^2_\text{Luft}(p)=&\SI{1336(4)e-10}{\milli\bar^{-1}}\cdot p + 1 + (10\,\pm\,30)\cdot 10^{-8} \\
  n^2_{\symup{CO}_2}(p)=&\SI{199(1)e-9}{\milli\bar^{-1}}\cdot p + 1 + (4\,\pm\,3)\cdot 10^{-6}
\end{align*}
%
Mit Hilfe der bestimmten Parameter und des zur Versuchsdurchführung herrschenden
Umgebungsdrucks und der Umgebungstemperatur lässt sich also der Brechungsindex der Gase
berechnen. Die nächstgelegene Wetterstation hat zur Versuchsdurchführung einen
Luftdruck von $\SI{1018}{\hecto\pascal}$~\cite{wetteronline} verzeichnet. Daraus folgt
für die Brechungsindices:
%1.0001358907668496
\begin{align*}
  n_\text{Luft}(p=\SI{1018}{\milli\bar})=& 1,000\,135\,9\,\pm\,0,000\,000\,3 \\
  n_{\symup{CO}_2}(p=\SI{1018}{\milli\bar})=& 1,000\,198\,\pm\,0,000\,003
\end{align*}
%
Nach Gleichung~\eqref{eq:näherung} ist der Proportionalitätsfaktor
der Druckabhängigkeit für Gase mit einem Brechungsindex nahe 1 etwa
$\frac{3A}{RT}$. Es gilt also
%
\begin{equation}
  a=\frac{3A}{RT},
\end{equation}
%
wobei $a$ die aus der Ausgleichsrechnung bestimmte Steigung ist, $A$ die molare Refraktivität des Mediums beschreibt, $R$ die universelle Gaskonstante und $T$ die Temperatur ist. Somit lässt sich der aus den Messdaten bestimmte Wert für $a$ auf Normalbedingungen umrechnen. Zur Berechnung der Brechungsindices unter Normalbedingungen ($T_0=\SI{15}{\degree\celsius}$, $p=\,1\,\text{atm}$) gilt:
%
\begin{align*}
  a_0=\frac{aT}{T_0} .
\end{align*}
%
Es folgt also für die Brechungsindices bei Normalbedingungen:
%
\begin{align*}
  n_\text{Luft}(p=\,1\,\text{atm})=& 1,000\,223\,7\,\pm\,0,000\,000\,2 \\
  n_{\symup{CO}_2}(p=\,1\,\text{atm})=& 1,000\,329\,\pm\,0,000\,003
\end{align*}
%

\begin{figure}[htb]
  \centering
  \includegraphics[width=.7\textwidth]{analysis/Luft1.pdf}
  \caption{Erste Messung des Brechungsindex von Luft. Aufgetragen sind die aus den Messwerten bestimmten Brechungsindices, sowie die Ausgleichsgerade.}
  \label{fig:luft1}
\end{figure}%

\begin{figure}
  \centering
  \includegraphics[width=.7\textwidth]{analysis/Luft2.pdf}
  \caption{Zweite Messung des Brechungsindex von Luft. Aufgetragen sind die aus den Messwerten bestimmten Brechungsindices, sowie die Ausgleichsgerade.}
  \label{fig:luft2}
\end{figure}

\begin{figure}
  \centering
  \includegraphics[width=.7\textwidth]{analysis/Luft3.pdf}
  \caption{Dritte Messung des Brechungsindex von Luft. Aufgetragen sind die aus den Messwerten bestimmten Brechungsindices, sowie die Ausgleichsgerade.}
  \label{fig:luft3}
\end{figure}%

\begin{figure}
  \centering
  \includegraphics[width=.7\linewidth]{analysis/CO2_1.pdf}
  \caption{Erste Messung des Brechungsindex von Kohlenstoffdioxid. Aufgetragen sind die aus den Messwerten bestimmten Brechungsindices, sowie die Ausgleichsgerade.}
  \label{fig:co2_1}
\end{figure}

\begin{figure}
  \centering
  \includegraphics[width=.7\linewidth]{analysis/CO2_2.pdf}
  \caption{Zweite Messung des Brechungsindex von Kohlenstoffdioxid. Aufgetragen sind die aus den Messwerten bestimmten Brechungsindices, sowie die Ausgleichsgerade.}
  \label{fig:co2_2}
\end{figure}

\begin{figure}
  \centering
  \includegraphics[width=.7\linewidth]{analysis/CO2_3.pdf}
  \caption{Dritte Messung des Brechungsindex von Kohlenstoffdioxid. Aufgetragen sind die aus den Messwerten bestimmten Brechungsindices, sowie die Ausgleichsgerade.}
  \label{fig:co2_3}
\end{figure}

\begin{table}
  \centering
  \begin{tabular}{c
                  S
                  S}
    \toprule
    {\textbf{Luft}} & {a/$\si{\milli\bar^{-1}}$} & {(b-1)/1} \\
    \midrule
    Messung 1 & 1330(8)e-10 & -3(5)e-7 \\
    Messung 2 & 1339(8)e-10 & 2(5)e-7 \\
    Messung 3 & 134(1)e-9 & 3(6)e-7 \\
    \midrule
    {$\symup{\mathbf{CO_2}}$} & {a/$\si{\milli\bar^{-1}}$} & {(b-1)/1} \\
    \midrule
    Messung 1 & 198(1)e-9 & 82(6)e-7 \\
    Messung 2 & 200(1)e-9 & 17(7)e-7 \\
    Messung 3 & 200(1)e-9 & 26(7)e-7 \\
    \bottomrule
  \end{tabular}
\caption{Ergebnisse für die Parameter der linearen Ausgleichsrechnung für die drei Messungen der jeweiligen Gase.}
  \label{tab:lin}
\end{table}
