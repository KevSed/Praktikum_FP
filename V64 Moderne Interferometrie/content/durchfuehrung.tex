\section{Durchführung}
\label{sec:durchführung}

\subsection{Versuchsaufbau}
Zunächst wird das Sagnac Interferometer, wie in
Abschnitt~\ref{sec:theorie_das_sagnac_interferometer} beschrieben, aufgebaut.
Dabei müssen die Spiegel mit Hilfe von sogenannten Strahlpaddeln präzise
justiert werden, um zu gewährleisten, dass die beiden entgegengesetzt laufenden
Strahlen im Interferometer an jedem Ort deckungsgleich aufeinanderliegen.
Anschließend wird durch vorsichtiges Verschieben des Spiegels vor dem Eingang
des Interferometers der Strahl im Interferometer aufgeteilt. Ist dies erreicht,
wird der Aufbau mit Hilfe einer durchsichtigen Kunstoffhaube geschützt.

Der am Interferometer austretende Strahl wird auf einen zweiten Strahlteiler
gelenkt, der den Strahl erneut aufteilt, wobei der reflektierte Strahl nun
um~$\SI{45}{\degree}$ aus der horizontalen Ebene herausgelenkt wird. Die
Intensitäten beider Teilstrahlen werden anschließend von je einer Photodiode
gemessen, die ihre Spannungssignale an eine automatische Zählapparatur
weitergeben.

\subsection{Versuchsdurchführung}
In einer ersten Messung wird der Kontrast des verwendeten Interferometers
bestimmt. Dazu wird vor dem Strahlteiler ein Polarisationsfilter platziert und
in~$\SI{15}{\degree}$ Schritten von~$\SI{0}{\degree}$ bis~$\SI{360}{\degree}$
variiert. Eine der beiden Dioden wird mit einem Amperemeter verbunden, welches
den detektierten Strom in Abhängigkeit vom eingestellten Polarisationswinkel
misst. Für jeden eingestellten Polarisationswinkel wird durch Drehung des
Doppelglashalters die minimale und die maximale Intensität gesucht. Für die
folgenden Versuchsteile wird stets die Konfiguration verwendet, die einem
maximalen Kontrast entspricht.

Die zweite Messung dient der Bestimmung des Brechungsindex von Glas. Dazu wird
der Winkel am Doppelglashalter zehnmal langsam von~$\SI{0}{\degree}$
auf~$\SI{10}{\degree}$ gedreht. Die Ausgänge der Photodioden wurden vorher mit
der Zählautomatik verbunden, die in der Lage ist durch Subtraktion der
Stromstärken und Zählung der Nulldurchgänge die Anzahl von Interferenzmaxima zu
zählen. Es werden zehnmal die Anzahl an gezählten Maxima abgelesen.

Im dritten Versuchsteil wird eine Gaszelle in einen der beiden Strahlgänge im
Interferometer eingebracht. Durch Evakuierung und Ventilierung der Gaszelle mit
Luft bzw. Kohlenstoffdioxid, werden je drei Messreihen in Abhängigkeit des
Drucks aufgenommen. Nach erfolgter Evakuierung wird die Gaszelle langsam wieder
befüllt und die Anzahl der Interferenzmaxima wird in~$\SI{50}{\milli\bar}$
abgelesen.
