\section{Theorie}
\label{sec:theorie}

\subsection{Das Sagnac Interferometer}
\label{sec:theorie_das_sagnac_interferometer}
Abbildung~\ref{fig:aufbau} zeigt den Aufbau des Sagnac Interferometers. Der
Strahl eines Helium-Neon-Lasers ($\lambda=\SI{632.8}{\nano\metre}$) wird über
zwei Spiegel auf den Eingang des Interferometers gelenkt. Dort trifft dieser
senkrecht auf einen polarisierenden Strahlteiler. Während ein Teil des Strahls
den Strahlteiler auf geradem Weg passiert, wird der andere Teil im rechten
Winkel reflektiert. Die zwei Teilstrahlen sind nach Austritt aus dem
Strahlteiler senkrecht zueinander polarisiert. Der Strahlteiler selbst besteht
aus zwei Prismen mit dreieckiger Grundfläche, die auf ihren Hypothenusen
miteinander verbunden sind. Der Normalenvektor der Hypothenusenfläche steht
im~$\SI{45}{\degree}$ Winkel zum einfallenden und reflektierten Strahl. Die zwei
Strahlen laufen nun über drei Spiegel in einem Rechteck, wobei sie sich dabei
überlagern. Danach treffen sie erneut auf den Strahlteiler und verlassen das
Interferometer in Richtung einer Messvorrichtung bestehend aus zwei Photodioden.

Durch vorsichtiges Verschieben des Spiegels vor dem Eingang des Interferometers,
kann der überlagerte Strahl im Interferometer in zwei entgegengesetzt laufende
Strahlen räumlich getrennt werden. Es besteht nun die Möglichkeit ein optisches
Element in einen der beiden Teilstrahlen zu stellen, ohne den anderen Teilstrahl
zu beeinflussen. Ein weiterer Vorteil des Sagnac Interferometers ist, dass es
unempfindlich gegenüber kleineren Erschütterungen ist, da beide Strahlen die
gleiche Wegstrecke durchlaufen.
\begin{figure}[htb]
  \centering
  \includegraphics[width=0.8\textwidth]{figures/aufbau.pdf}
  \caption{Aufbau des verwendeten Sagnac Interferometers~\cite{V64}.}
  \label{fig:aufbau}
\end{figure}

\subsection{Kontrast des Sagnac Interferometers}
Der Kontrast eines Interferometers ist gegeben durch
\begin{equation}
  K=\frac{I_{\text{max}}-I_{\text{min}}}{I_{\text{max}}+I_{\text{min}}}.
  \label{eq:kontrast}
\end{equation}
Hierbei bezeichnet~$I_{\text{max}}$ die Intensität eines Interferenzmaximums
und~$I_{\text{min}}$ die Intensität eines Interferenzminimums. Im Idealfall
ist~$I_{\text{min}}=0$ und somit~$K=1$. Im schlechtesten Fall ist kein
Unterschied in der Intensität zwischen Maxima und Minima zu erkennen. Dann
gilt~$I_{\text{max}}=I_{\text{min}}$ und somit~$K=0$.

Im durchgeführten Experiment wird der Kontrast in Abhängigkeit der Einstellung
eines im Strahlgang befindlichen Polarisationsfilters gemessen. Ausgedrückt
durch einen Polarisationswinkel~$\varphi$, einem Phasenwinkel~$\delta$ und einer
Amplitude~$A$ beträgt der Kontrast
\begin{equation}
  K=A\lvert\sin\left(2\varphi+\delta\right)\rvert,
  \label{eq:kontrast}
\end{equation}

\subsection{Bestimmung des Brechungsindex eines Gases}
Der Brechungsindex~$n$ von Materie ist definiert als Quotient der
Lichtgeschwindigkeit im Vakuum und der Lichtgeschwindigkeit in der Materie.
Daraus folgt für die Wellenzahl~$k$ des Lichts in Materie
\begin{equation}
  k=\frac{2\pi}{\lambda_{\text{vac}}}n
\end{equation}
wobei~$\lambda_{\text{vac}}$ die Wellenlänge des Lichts im Vakuum ist.

Für die Bestimmung des Brechungsindex eines Gases wird nun eine Gaszelle der
Länge~$L$ in einem(!) Strahl des Interferometers platziert. Durch die leicht
unterschiedliche Geschwindigkeit des Lichts in der Gaszelle, kommt es zu einer
Phasendifferenz zwischen den beiden Strahlen
\begin{equation}
  \Delta\varphi=\frac{2\pi L}{\lambda_{\text{vac}}}\Delta n,
\end{equation}
die linear mit der Differenz der Brechungsindices des Gases und der
Umgebungsluft zusammenhängt. Durch Überlagerung der beiden Strahlen ergibt sich
ein Interferenzmuster. Die Anzahl~$M$ der beobachteten Interferenzmaxima ist
proportional zur Phasendifferenz. Es gilt
\begin{gather}
  M=\frac{\Delta\varphi}{2\pi}=\frac{L}{\lambda_{\text{vac}}}(n-1) \\
  \shortintertext{und somit}
  n=\frac{\lambda_{\text{vac}}}{L}M+1.
  \label{eq:brechungsindex_gas}
\end{gather}
wobei die Annahme gemacht wurde, dass für den Brechungsindex der
Umgebungsluft~$n_{\text{Luft}}=1$ gilt. Im Allgemeinen ist der Brechungsindex
temperatur- und druckabhängig. Einen Zusammenhang liefert die
Lorentz-Lorenz-Gleichung
\begin{equation}
  \frac{n^2-1}{n^2+2}=\frac{4\pi}{3}N\alpha=\frac{Ap}{RT}
\end{equation}
mit der Polarisierbarkeit~$\alpha$, der Anzahl~$N$ der Moleküle im
Einheitsvolumen, der molaren Refraktivität~$A$, der universellen
Gaskonstante~$R$, sowie Druck~$p$ und Temperatur~$T$. Für Gase liegt der
Brechungsindex in der Regel nahe Eins, weswegen~$n^2+2\approx3$ gilt. Damit
ergibt sich in guter Näherung
\begin{equation}
  n\approx\sqrt{1+\frac{3Ap}{RT}}
  \label{eq:näherung}
\end{equation}

\subsection{Bestimmung des Brechungsindex eines lichtdurchlässigen Festkörpers}
Während die Bestimmung des Brechungsindex eines Gases mit Hilfe der Variation
des Gasdrucks und der Temperatur erfolgen kann, so ist dies für Festkörper nicht
ohne Weiteres praktikabel. Ein im Folgenden diskutiertes Verfahren ermöglicht
die Bestimmung des Brechungsindexes von lichtdurchlässigen, planparallelen
Festkörpern, wie zum Beispiel eines kleinen Glasplättchens.

\begin{figure}[htb]
  \centering
  \includegraphics[width=0.8\textwidth]{figures/glasplättchen.pdf}
  \caption{Schematischer Strahlengang des Lichtes an einem Glasplättchen.}
  \label{fig:plättchen}
\end{figure}

Abbildung~\ref{fig:plättchen} zeigt schematisch den Strahlengang des Lichtes in
einem Glasplättchen der Dicke~$T$. Hierbei wird deutlich, dass die Brechung des
Lichtes eine Trajektorie $\overline{CD}$ verursacht, die im Vergleich zum
direkten Weg $\overline{AB}$ länger ist. Dies verursacht den zur Interferenz
benötigten Phasenunterschied. Dieser berechnet sich über den Zusammenhang
\begin{equation}
  \Delta\phi=\frac{2\pi}{\lambda_{\text{vac}}}\left(n_{\text{Glas}}\cdot
  \overline{CD}-n_{\text{Luft}}\cdot\overline{AB}\right).
\end{equation}
Er hängt damit von der effektiven Weglänge des Lichtes ab, die wiederum abhängig
vom Brechungsindex ist. Aus Abbildung~\ref{fig:plättchen} folgen für
$\overline{AB}$ und $\overline{CD}$ die geometrischen Zusammenhänge
\begin{align}
  \cos\theta'=&\frac{T}{\overline{CD}}\quad\Rightarrow\quad\overline{CD}
  =\frac{T}{\cos\theta'} \\
  \cos(\theta-\theta')=&\frac{\overline{AB}}{\overline{CD}}
\end{align}
Der Phasenunterschied berechnet sich dann für $n_{\text{Luft}}=1$ und
$n_{\text{Glas}}=n$ zu
\begin{align*}
  \Delta\phi=&\frac{2\pi}{\lambda_{\text{vac}}}(n\cdot\overline{CD}
  -\overline{AB}) \\
  =&\frac{2\pi}{\lambda_{\text{vac}}}\left(n\frac{T}{\cos{\theta'}}
  -\overline{CD}\cos(\theta-\theta')\right) \\
  =&\frac{2\pi T}{\lambda_{\text{vac}}}\left(\frac{n-\cos(\theta-\theta')}
  {\cos\theta'}\right)
\end{align*}
Mit Hilfe des Brechungsgesetzes von Snellius lässt sich nun der Winkel~$\theta'$
durch den Winkel~$\theta$ ausdrücken. Es gilt demnach
\begin{equation}
  n\cdot\sin\theta'=1\cdot\sin\theta\quad\Rightarrow\quad\theta'
  =\arcsin\left(\frac{\sin\theta}{n}\right)
\end{equation}
Im Versuchsaufbau wird anstatt eines einzelnen Glasplättchens ein
Doppelglashalter verwendet. Dabei handelt es sich um zwei(!) Glasplättchen, die
sich mit einem festen Winkel von~$\SI{20}{\degree}$ zueinander in jeweils einem
Teilstrahl befinden und dort gemeinsam rotieren können. Symmetrisiert man das
System gelten die obigen Formeln
mit~$\theta\rightarrow\theta_{\pm}=\theta\pm\theta_0$
und~$\theta_0=\SI{10}{\degree}$ für zwei einzelne Glasplättchen, wobei die
jeweils berechneten Phasenunterschiede aufgrund der unterschiedlichen
Lichtlaufrichtung voneinander subtrahiert werden müssen. Für die Anzahl~$M$ der
beobachteten Interferenzmaxima gilt dann
\begin{align}
  M&=\frac{\Delta\Phi}{2\pi} \\
  &=\frac{T}{\lambda_{\text{vac}}}\left(\frac{n-\cos\left(\theta_{+}
  -\arcsin\left(\frac{\sin\theta_{+}}{n}\right)\right)}{\cos\left(\arcsin\left(
  \frac{\sin\theta_{+}}{n}\right)\right)}-\frac{n-\cos\left(\theta_{-}-\arcsin
  \left(\frac{\sin\theta_{-}}{n}\right)\right)}{\cos\left(\arcsin\left(
  \frac{\sin\theta_{-}}{n}\right)\right)}\right)
\end{align}
wobei nur noch der Drehwinkel~$\theta$ eine unabhängige Variable ist.
