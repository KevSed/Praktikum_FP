Die in der Auswertung verwendeten Mittelwerte mehrfach gemessener Größen sind gemäß der Gleichung
%
\begin{equation}
    \bar{x}=\frac{1}{n}\sum_{i=1}^n x_i
    \label{eq:mittelwert}
\end{equation}
%
bestimmt.
Die Standardabweichung des Mittelwertes ergibt sich dabei zu
%
\begin{equation}
    \mathup{\Delta}\bar{x}=\sqrt{\frac{1}{n(n-1)}\sum_{i=1}^n\left(x_i-\bar{x}\right)^2}.
    \label{eq:standardabweichung}
\end{equation}
%
Resultiert eine Größe über eine Gleichung aus zwei anderen fehlerbehafteten Größen, so berechnet sich der Gesamtfehler nach der Gaußschen Fehlerfortpflanzung zu
%
\begin{equation}
    \mathup{\Delta}f(x_1,x_2,...,x_n)=\sqrt{\left(\frac{\partial f}{\partial x_1}\mathup{\Delta}x_1\right)^2+\left(\frac{\partial f}{\partial x_2}\mathup{\Delta}x_2\right)^2+ \dotsb +\left(\frac{\partial f}{\partial x_n}\mathup{\Delta}x_n\right)^2}.
    \label{eq:fehlerfortpflanzung}
\end{equation}
%
Alle in der Auswertung angegebenen Größen sind stets auf die erste signifikante Stelle des Fehlers gerundet.
Setzt sich eine Größe über mehrere Schritte aus anderen Größen zusammen, so wird erst am Ende gerundet, um Fehler zu vermeiden.
Zur Auswertung wird die Programmiersprache \texttt{python (Version 3.4.1)}
mit den Bibliothekserweiterungen \texttt{numpy}, \texttt{scipy} und \texttt{matplotlib} zur Erstellung der Grafiken und linearen Regressionen verwendet.